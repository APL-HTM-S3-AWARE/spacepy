%
% API Documentation for SpacePy: Space Science Tools for Python
% Module spacepy.poppy
%
% Generated by epydoc 3.0.1
% [Mon May 17 14:36:32 2010]
%

%%%%%%%%%%%%%%%%%%%%%%%%%%%%%%%%%%%%%%%%%%%%%%%%%%%%%%%%%%%%%%%%%%%%%%%%%%%
%%                          Module Description                           %%
%%%%%%%%%%%%%%%%%%%%%%%%%%%%%%%%%%%%%%%%%%%%%%%%%%%%%%%%%%%%%%%%%%%%%%%%%%%

    \index{spacepy \textit{(package)}!spacepy.poppy \textit{(module)}|(}
\section{Module spacepy.poppy}

    \label{spacepy:poppy}
PoPPy -- Point Processes in Python.

This module contains point process class types and a variety of functions 
for association analysis. The routines given here grew from work presented 
by Morley and Freeman (Geophysical Research Letters, 34, L08104, 
doi:10.1029/ 2006GL028891, 2007), which were originally written in IDL. 
This module is intended for application to discrete time series of events 
to assess statistical association between the series and to calculate 
confidence limits. Any mis-application or mis-interpretation by the user is
the user's own fault.

Each instance must be initialized with:

\begin{alltt}
\pysrcprompt{{\textgreater}{\textgreater}{\textgreater} }obj = poppy.PPro(series1, series2)\end{alltt}
To perform association analysis

\begin{alltt}
\pysrcprompt{{\textgreater}{\textgreater}{\textgreater} }obj.assoc(u=lags, h=halfwindow)\end{alltt}
To plot

\begin{alltt}
\pysrcprompt{{\textgreater}{\textgreater}{\textgreater} }obj.aaplot()\end{alltt}
--++-- By Steve Morley --++--

smorley@lanl.gov/morley\_steve@hotmail.com, Los Alamos National Laboratory,
ISR-1, PO Box 1663, Los Alamos, NM 87545

\textbf{Author:} Steve Morley (smorley@lanl.com/morley\_steve@hotmail.com)




%%%%%%%%%%%%%%%%%%%%%%%%%%%%%%%%%%%%%%%%%%%%%%%%%%%%%%%%%%%%%%%%%%%%%%%%%%%
%%                               Functions                               %%
%%%%%%%%%%%%%%%%%%%%%%%%%%%%%%%%%%%%%%%%%%%%%%%%%%%%%%%%%%%%%%%%%%%%%%%%%%%

  \subsection{Functions}

    \label{spacepy:poppy:boots_ci}
    \index{spacepy \textit{(package)}!spacepy.poppy \textit{(module)}!spacepy.poppy.boots\_ci \textit{(function)}}

    \vspace{0.5ex}

\hspace{.8\funcindent}\begin{boxedminipage}{\funcwidth}

    \raggedright \textbf{boots\_ci}(\textit{data}, \textit{n}, \textit{inter}, \textit{func})

    \vspace{-1.5ex}

    \rule{\textwidth}{0.5\fboxrule}
\setlength{\parskip}{2ex}
    Construct bootstrap confidence interval - caution: slow!

    Input: n is number of surrogates; data is data array (1D); inter is 
    desired confidence interval (e.g. 95\%); func is a user-defined 
    function (lambda)

    Example:

\begin{alltt}
\pysrcprompt{{\textgreater}{\textgreater}{\textgreater} }data, n = numpy.random.lognormal(mean=5.1, sigma=0.3, size=3000), 4000.
\pysrcprompt{{\textgreater}{\textgreater}{\textgreater} }myfunc = \pysrckeyword{lambda} x: numpy.median(x)
\pysrcprompt{{\textgreater}{\textgreater}{\textgreater} }ci\_low, ci\_high = poppy.boots\_ci(data, n, 95, myfunc)
\pysrcprompt{{\textgreater}{\textgreater}{\textgreater} }ci\_low, numpy.median(data), ci\_high
\pysrcoutput{(163.96354196633686, 165.2393331896551, 166.60491435416566) iter. 1}
\pysrcoutput{... repeat}
\pysrcoutput{(162.50379144492726, 164.15218265100233, 165.42840588032755) iter. 2}\end{alltt}
    For comparison:

\begin{alltt}
\pysrcprompt{{\textgreater}{\textgreater}{\textgreater} }data = numpy.random.lognormal(mean=5.1, sigma=0.3, size=90000)
\pysrcprompt{{\textgreater}{\textgreater}{\textgreater} }numpy.median(data)
\pysrcoutput{163.83888237895815}\end{alltt}
    Note that the true value of the desired quantity may lie outside the 
    95\% confidence interval one time in 20 realizations. This occurred for
    the first iteration - please feel free to play with this.

\setlength{\parskip}{1ex}
    \end{boxedminipage}


%%%%%%%%%%%%%%%%%%%%%%%%%%%%%%%%%%%%%%%%%%%%%%%%%%%%%%%%%%%%%%%%%%%%%%%%%%%
%%                               Variables                               %%
%%%%%%%%%%%%%%%%%%%%%%%%%%%%%%%%%%%%%%%%%%%%%%%%%%%%%%%%%%%%%%%%%%%%%%%%%%%

  \subsection{Variables}

    \vspace{-1cm}
\hspace{\varindent}\begin{longtable}{|p{\varnamewidth}|p{\vardescrwidth}|l}
\cline{1-2}
\cline{1-2} \centering \textbf{Name} & \centering \textbf{Description}& \\
\cline{1-2}
\endhead\cline{1-2}\multicolumn{3}{r}{\small\textit{continued on next page}}\\\endfoot\cline{1-2}
\endlastfoot\raggedright \_\-\_\-p\-a\-c\-k\-a\-g\-e\-\_\-\_\- & \raggedright \textbf{Value:} 
{\tt \texttt{'}\texttt{spacepy}\texttt{'}}&\\
\cline{1-2}
\end{longtable}


%%%%%%%%%%%%%%%%%%%%%%%%%%%%%%%%%%%%%%%%%%%%%%%%%%%%%%%%%%%%%%%%%%%%%%%%%%%
%%                           Class Description                           %%
%%%%%%%%%%%%%%%%%%%%%%%%%%%%%%%%%%%%%%%%%%%%%%%%%%%%%%%%%%%%%%%%%%%%%%%%%%%

    \index{spacepy \textit{(package)}!spacepy.poppy \textit{(module)}!spacepy.poppy.PPro \textit{(class)}|(}
\subsection{Class PPro}

    \label{spacepy:poppy:PPro}
\begin{tabular}{cccccc}
% Line for object, linespec=[False]
\multicolumn{2}{r}{\settowidth{\BCL}{object}\multirow{2}{\BCL}{object}}
&&
  \\\cline{3-3}
  &&\multicolumn{1}{c|}{}
&&
  \\
&&\multicolumn{2}{l}{\textbf{spacepy.poppy.PPro}}
\end{tabular}

PoPPy point process object

Initialize object with series1 and series2. These should be timeseries of 
events, given as lists, arrays, or lists of datetime objects. Includes 
method to perform association analysis of input series

Output can be nicely plotted with plot method


%%%%%%%%%%%%%%%%%%%%%%%%%%%%%%%%%%%%%%%%%%%%%%%%%%%%%%%%%%%%%%%%%%%%%%%%%%%
%%                                Methods                                %%
%%%%%%%%%%%%%%%%%%%%%%%%%%%%%%%%%%%%%%%%%%%%%%%%%%%%%%%%%%%%%%%%%%%%%%%%%%%

  \subsubsection{Methods}

    \vspace{0.5ex}

\hspace{.8\funcindent}\begin{boxedminipage}{\funcwidth}

    \raggedright \textbf{\_\_init\_\_}(\textit{self}, \textit{process1}, \textit{process2}, \textit{lags}={\tt None}, \textit{winhalf}={\tt None})

\setlength{\parskip}{2ex}
    x.\_\_init\_\_(...) initializes x; see x.\_\_class\_\_.\_\_doc\_\_ for 
    signature

\setlength{\parskip}{1ex}
      Overrides: object.\_\_init\_\_ 	extit{(inherited documentation)}

    \end{boxedminipage}

    \vspace{0.5ex}

\hspace{.8\funcindent}\begin{boxedminipage}{\funcwidth}

    \raggedright \textbf{\_\_str\_\_}(\textit{self})

    \vspace{-1.5ex}

    \rule{\textwidth}{0.5\fboxrule}
\setlength{\parskip}{2ex}
    String Representation of PoPPy object

\setlength{\parskip}{1ex}
      Overrides: object.\_\_str\_\_

    \end{boxedminipage}

    \vspace{0.5ex}

\hspace{.8\funcindent}\begin{boxedminipage}{\funcwidth}

    \raggedright \textbf{\_\_repr\_\_}(\textit{self})

    \vspace{-1.5ex}

    \rule{\textwidth}{0.5\fboxrule}
\setlength{\parskip}{2ex}
    String Representation of PoPPy object

\setlength{\parskip}{1ex}
      Overrides: object.\_\_repr\_\_

    \end{boxedminipage}

    \label{spacepy:poppy:PPro:__len__}
    \index{spacepy \textit{(package)}!spacepy.poppy \textit{(module)}!spacepy.poppy.PPro \textit{(class)}!spacepy.poppy.PPro.\_\_len\_\_ \textit{(method)}}

    \vspace{0.5ex}

\hspace{.8\funcindent}\begin{boxedminipage}{\funcwidth}

    \raggedright \textbf{\_\_len\_\_}(\textit{self})

    \vspace{-1.5ex}

    \rule{\textwidth}{0.5\fboxrule}
\setlength{\parskip}{2ex}
    Calling len(obj) will return the number of points in process 1

\setlength{\parskip}{1ex}
    \end{boxedminipage}

    \label{spacepy:poppy:PPro:swap}
    \index{spacepy \textit{(package)}!spacepy.poppy \textit{(module)}!spacepy.poppy.PPro \textit{(class)}!spacepy.poppy.PPro.swap \textit{(method)}}

    \vspace{0.5ex}

\hspace{.8\funcindent}\begin{boxedminipage}{\funcwidth}

    \raggedright \textbf{swap}(\textit{self})

    \vspace{-1.5ex}

    \rule{\textwidth}{0.5\fboxrule}
\setlength{\parskip}{2ex}
    Swaps process 1 and process 2

\setlength{\parskip}{1ex}
    \end{boxedminipage}

    \label{spacepy:poppy:PPro:assoc}
    \index{spacepy \textit{(package)}!spacepy.poppy \textit{(module)}!spacepy.poppy.PPro \textit{(class)}!spacepy.poppy.PPro.assoc \textit{(method)}}

    \vspace{0.5ex}

\hspace{.8\funcindent}\begin{boxedminipage}{\funcwidth}

    \raggedright \textbf{assoc}(\textit{self}, \textit{u}={\tt None}, \textit{h}={\tt None})

    \vspace{-1.5ex}

    \rule{\textwidth}{0.5\fboxrule}
\setlength{\parskip}{2ex}
    Perform association analysis on input series

    u = range of lags h = association window half-width

\setlength{\parskip}{1ex}
    \end{boxedminipage}

    \label{spacepy:poppy:PPro:plot}
    \index{spacepy \textit{(package)}!spacepy.poppy \textit{(module)}!spacepy.poppy.PPro \textit{(class)}!spacepy.poppy.PPro.plot \textit{(method)}}

    \vspace{0.5ex}

\hspace{.8\funcindent}\begin{boxedminipage}{\funcwidth}

    \raggedright \textbf{plot}(\textit{self}, \textit{figsize}={\tt None}, \textit{dpi}={\tt 300})

    \vspace{-1.5ex}

    \rule{\textwidth}{0.5\fboxrule}
\setlength{\parskip}{2ex}
    Method called to create basic plot of association analysis.

    Inputs: Uses object attributes created by the obj.assoc() method.

    Optional keyword(s): usrlimy (default = []) - override automatic 
    y-limits on plot.

\setlength{\parskip}{1ex}
    \end{boxedminipage}

    \label{spacepy:poppy:PPro:aa_ci}
    \index{spacepy \textit{(package)}!spacepy.poppy \textit{(module)}!spacepy.poppy.PPro \textit{(class)}!spacepy.poppy.PPro.aa\_ci \textit{(method)}}

    \vspace{0.5ex}

\hspace{.8\funcindent}\begin{boxedminipage}{\funcwidth}

    \raggedright \textbf{aa\_ci}(\textit{self}, \textit{inter}, \textit{n\_boots}={\tt 1000})

    \vspace{-1.5ex}

    \rule{\textwidth}{0.5\fboxrule}
\setlength{\parskip}{2ex}
    Get bootstrap confidence intervals for association number

    Requires input of desired confidence interval, e.g.,

\begin{alltt}
\pysrcprompt{{\textgreater}{\textgreater}{\textgreater} }obj.aa\_ci(95)\end{alltt}
    Upper and lower confidence limits are added to the ci attribute

\setlength{\parskip}{1ex}
    \end{boxedminipage}


\large{\textbf{\textit{Inherited from object}}}

\begin{quote}
\_\_delattr\_\_(), \_\_format\_\_(), \_\_getattribute\_\_(), \_\_hash\_\_(), \_\_new\_\_(), \_\_reduce\_\_(), \_\_reduce\_ex\_\_(), \_\_setattr\_\_(), \_\_sizeof\_\_(), \_\_subclasshook\_\_()
\end{quote}

%%%%%%%%%%%%%%%%%%%%%%%%%%%%%%%%%%%%%%%%%%%%%%%%%%%%%%%%%%%%%%%%%%%%%%%%%%%
%%                              Properties                               %%
%%%%%%%%%%%%%%%%%%%%%%%%%%%%%%%%%%%%%%%%%%%%%%%%%%%%%%%%%%%%%%%%%%%%%%%%%%%

  \subsubsection{Properties}

    \vspace{-1cm}
\hspace{\varindent}\begin{longtable}{|p{\varnamewidth}|p{\vardescrwidth}|l}
\cline{1-2}
\cline{1-2} \centering \textbf{Name} & \centering \textbf{Description}& \\
\cline{1-2}
\endhead\cline{1-2}\multicolumn{3}{r}{\small\textit{continued on next page}}\\\endfoot\cline{1-2}
\endlastfoot\multicolumn{2}{|l|}{\textit{Inherited from object}}\\
\multicolumn{2}{|p{\varwidth}|}{\raggedright \_\_class\_\_}\\
\cline{1-2}
\end{longtable}

    \index{spacepy \textit{(package)}!spacepy.poppy \textit{(module)}!spacepy.poppy.PPro \textit{(class)}|)}
    \index{spacepy \textit{(package)}!spacepy.poppy \textit{(module)}|)}
