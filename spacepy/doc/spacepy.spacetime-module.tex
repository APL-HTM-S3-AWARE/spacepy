%
% API Documentation for SpacePy: Space Science Tools for Python
% Module spacepy.spacetime
%
% Generated by epydoc 3.0.1
% [Mon May 17 14:36:32 2010]
%

%%%%%%%%%%%%%%%%%%%%%%%%%%%%%%%%%%%%%%%%%%%%%%%%%%%%%%%%%%%%%%%%%%%%%%%%%%%
%%                          Module Description                           %%
%%%%%%%%%%%%%%%%%%%%%%%%%%%%%%%%%%%%%%%%%%%%%%%%%%%%%%%%%%%%%%%%%%%%%%%%%%%

    \index{spacepy \textit{(package)}!spacepy.spacetime \textit{(module)}|(}
\section{Module spacepy.spacetime}

    \label{spacepy:spacetime}
Implementation of TickTock class and SpaCo class functions

\textbf{Version:} \$Revision: 1.1.1.1 $, \$Date: 2010/05/20 17:19:45 $



\textbf{Author:} Josef Koller, Los Alamos National Lab (jkoller@lanl.gov)




%%%%%%%%%%%%%%%%%%%%%%%%%%%%%%%%%%%%%%%%%%%%%%%%%%%%%%%%%%%%%%%%%%%%%%%%%%%
%%                               Functions                               %%
%%%%%%%%%%%%%%%%%%%%%%%%%%%%%%%%%%%%%%%%%%%%%%%%%%%%%%%%%%%%%%%%%%%%%%%%%%%

  \subsection{Functions}

    \label{spacepy:spacetime:doy2date}
    \index{spacepy \textit{(package)}!spacepy.spacetime \textit{(module)}!spacepy.spacetime.doy2date \textit{(function)}}

    \vspace{0.5ex}

\hspace{.8\funcindent}\begin{boxedminipage}{\funcwidth}

    \raggedright \textbf{doy2date}(\textit{year}, \textit{doy}, \textit{dtobj}={\tt False})

    \vspace{-1.5ex}

    \rule{\textwidth}{0.5\fboxrule}
\setlength{\parskip}{2ex}
    convert day-of-year doy into a month and day after 
    http://pleac.sourceforge.net/pleac\_python/datesandtimes.html

    (section) Input:

      \begin{itemize}
      \setlength{\parskip}{0.6ex}
        \item year (int or array of int) : year

        \item doy (int or array of int) : day of year

      \end{itemize}

    (section) Returns:

      \begin{itemize}
      \setlength{\parskip}{0.6ex}
        \item month (int or array of int) : month as integer number

        \item day (int or array of int) : as integer number

      \end{itemize}

    (section) Example:

\begin{alltt}
\pysrcprompt{{\textgreater}{\textgreater}{\textgreater} }month, day = doy2date(2002, 186)
\pysrcprompt{{\textgreater}{\textgreater}{\textgreater} }dts = doy2date([2002]*4, range(186,190), dtobj=True)\end{alltt}
    (section) See Also:

      getDOY

    (section) Author:

      Josef Koller, Los Alamos National Lab, jkoller@lanl.gov

    (section) Version:

      V1: 24-Jan-2010: can handle arrays as input (JK) V2: 02-Apr-2010: 
      option to return date objects (SM) V3: 07-Apr-2010: modified to 
      return datetime objects (SM)

\setlength{\parskip}{1ex}
    \end{boxedminipage}

    \label{spacepy:spacetime:tickrange}
    \index{spacepy \textit{(package)}!spacepy.spacetime \textit{(module)}!spacepy.spacetime.tickrange \textit{(function)}}

    \vspace{0.5ex}

\hspace{.8\funcindent}\begin{boxedminipage}{\funcwidth}

    \raggedright \textbf{tickrange}(\textit{start}, \textit{end}, \textit{deltadays}, \textit{dtype}={\tt \texttt{'}\texttt{ISO}\texttt{'}})

    \vspace{-1.5ex}

    \rule{\textwidth}{0.5\fboxrule}
\setlength{\parskip}{2ex}
    return a TickTock range given the start, end, and delta

    (section) Input:

      \begin{itemize}
      \setlength{\parskip}{0.6ex}
        \item start (string or number) : start time

        \item end (string or number) : end time (inclusive)

        \item deltadays: step in units of days (float); or datetime timedelta 
          object

        \item (optional) dtype (string) : data type for start, end; e.g. ISO, 
          UTC, RTD, etc. see TickTock for all options

      \end{itemize}

    (section) Returns:

      \begin{itemize}
      \setlength{\parskip}{0.6ex}
        \item ticks (TickTock instance)

      \end{itemize}

    (section) Example:

\begin{alltt}
\pysrcprompt{{\textgreater}{\textgreater}{\textgreater} }ticks = st.tickrange(\pysrcstring{'2002-02-01T00:00:00'}, \pysrcstring{'2002-02-10T00:00:00'}, deltadays = 1)
\pysrcprompt{{\textgreater}{\textgreater}{\textgreater} }ticks
\pysrcoutput{TickTock( ['2002-02-01T00:00:00', '2002-02-02T00:00:00', '2002-02-03T00:00:00', }
\pysrcoutput{'2002-02-04T00:00:00'] ), dtype=ISO}\end{alltt}
    (section) See Also:

      TickTock

    (section) Author:

      Josef Koller, Los Alamos National Lab, jkoller@lanl.gov

    (section) Version:

      V1: 10-Mar-2010: (JK) V1.1: 16-Mar-2010: fixed bug with floating 
      point precision (JK) V1.2: 28-Apr-2010: added timedelta support for 
      increment (SM)

    (section) Notes:

      This, and TickTock, do not currently support sub-second precision. 
      This should be fixed.

\setlength{\parskip}{1ex}
    \end{boxedminipage}


%%%%%%%%%%%%%%%%%%%%%%%%%%%%%%%%%%%%%%%%%%%%%%%%%%%%%%%%%%%%%%%%%%%%%%%%%%%
%%                               Variables                               %%
%%%%%%%%%%%%%%%%%%%%%%%%%%%%%%%%%%%%%%%%%%%%%%%%%%%%%%%%%%%%%%%%%%%%%%%%%%%

  \subsection{Variables}

    \vspace{-1cm}
\hspace{\varindent}\begin{longtable}{|p{\varnamewidth}|p{\vardescrwidth}|l}
\cline{1-2}
\cline{1-2} \centering \textbf{Name} & \centering \textbf{Description}& \\
\cline{1-2}
\endhead\cline{1-2}\multicolumn{3}{r}{\small\textit{continued on next page}}\\\endfoot\cline{1-2}
\endlastfoot\raggedright \_\-\_\-l\-o\-g\-\_\-\_\- & \raggedright \textbf{Value:} 
{\tt \texttt{'}\texttt{{\textbackslash}n\$Log: spacepy.spacetime-module.tex,v $
{\tt \texttt{'}\texttt{{\textbackslash}n\Revision 1.1.1.1  2010/05/20 17:19:45  smorley
{\tt \texttt{'}\texttt{{\textbackslash}n\Pre-release repo import for SpacePy
{\tt \texttt{'}\texttt{{\textbackslash}n\
{\tt \texttt{'}\texttt{{\textbackslash}n\Revision 1.1  2010/05/19 22:30:51  smorley
{\tt \texttt{'}\texttt{{\textbackslash}n\Regenerated documentation
{\tt \texttt{'}\texttt{{\textbackslash}n\{\textbackslash}nRevision 1.14  2010/05/17 17:3}\texttt{...}}&\\
\cline{1-2}
\raggedright \_\-\_\-p\-a\-c\-k\-a\-g\-e\-\_\-\_\- & \raggedright \textbf{Value:} 
{\tt \texttt{'}\texttt{spacepy}\texttt{'}}&\\
\cline{1-2}
\end{longtable}


%%%%%%%%%%%%%%%%%%%%%%%%%%%%%%%%%%%%%%%%%%%%%%%%%%%%%%%%%%%%%%%%%%%%%%%%%%%
%%                           Class Description                           %%
%%%%%%%%%%%%%%%%%%%%%%%%%%%%%%%%%%%%%%%%%%%%%%%%%%%%%%%%%%%%%%%%%%%%%%%%%%%

    \index{spacepy \textit{(package)}!spacepy.spacetime \textit{(module)}!spacepy.spacetime.SpaCo \textit{(class)}|(}
\subsection{Class SpaCo}

    \label{spacepy:spacetime:SpaCo}
\begin{tabular}{cccccc}
% Line for object, linespec=[False]
\multicolumn{2}{r}{\settowidth{\BCL}{object}\multirow{2}{\BCL}{object}}
&&
  \\\cline{3-3}
  &&\multicolumn{1}{c|}{}
&&
  \\
&&\multicolumn{2}{l}{\textbf{spacepy.spacetime.SpaCo}}
\end{tabular}

a = Spaco( data, dtype, carsph, [units, ticktock] )

A class holding spatial coordinates in cartesian/spherical in units of Re 
and degrees

(section) Input:

  \begin{itemize}
  \setlength{\parskip}{0.6ex}
    \item data (list or ndarray, dim = (n,3) ) : coordinate points

    \item dtype (string) :coordinate system, possible are GDZ, GEO, GSM, GSE, 
      SM, GEI MAG, SPH, RLL

    \item carsph (string) : cartesian or spherical, 'car' or 'sph'

    \item optional units (list of strings) : standard are  ['Re', 'Re', 'Re'] 
      or ['Re', 'deg', 'deg'] depending on the carsph content

    \item optional ticktock (TickTock instance) : used for coordinate 
      transformations (see a.convert)

  \end{itemize}

(section) Returns:

  \begin{itemize}
  \setlength{\parskip}{0.6ex}
    \item instance with a.data, a.carsph, etc.

  \end{itemize}

(section) Example:

\begin{alltt}
\pysrcprompt{{\textgreater}{\textgreater}{\textgreater} }coord = SpaCo([[1,2,4],[1,2,2]], \pysrcstring{'GEO'}, \pysrcstring{'car'})
\pysrcprompt{{\textgreater}{\textgreater}{\textgreater} }coord.x  \pysrccomment{\# returns all x coordinates}
\pysrcoutput{array([1, 1])}
\pysrcoutput{}\pysrcprompt{{\textgreater}{\textgreater}{\textgreater} }coord.ticktock = TickTock([\pysrcstring{'2002-02-02T12:00:00'}, \pysrcstring{'2002-02-02T12:00:00'}], \pysrcstring{'ISO'}) \pysrccomment{\# add ticktock}
\pysrcprompt{{\textgreater}{\textgreater}{\textgreater} }newcoord = coord.convert(\pysrcstring{'GSM'}, \pysrcstring{'sph'})
\pysrcprompt{{\textgreater}{\textgreater}{\textgreater} }newcoord\end{alltt}
(section) See Also:

  spacetime.TickTock class

(section) Author:

  Josef Koller, Los Alamos National Lab (jkoller@lanl.gov)

(section) Version:

  V1: 05-Mar-2010 (JK)


%%%%%%%%%%%%%%%%%%%%%%%%%%%%%%%%%%%%%%%%%%%%%%%%%%%%%%%%%%%%%%%%%%%%%%%%%%%
%%                                Methods                                %%
%%%%%%%%%%%%%%%%%%%%%%%%%%%%%%%%%%%%%%%%%%%%%%%%%%%%%%%%%%%%%%%%%%%%%%%%%%%

  \subsubsection{Methods}

    \vspace{0.5ex}

\hspace{.8\funcindent}\begin{boxedminipage}{\funcwidth}

    \raggedright \textbf{\_\_init\_\_}(\textit{self}, \textit{data}, \textit{dtype}, \textit{carsph}, \textit{units}={\tt None}, \textit{ticktock}={\tt None})

\setlength{\parskip}{2ex}
    x.\_\_init\_\_(...) initializes x; see x.\_\_class\_\_.\_\_doc\_\_ for 
    signature

\setlength{\parskip}{1ex}
      Overrides: object.\_\_init\_\_ 	extit{(inherited documentation)}

    \end{boxedminipage}

    \vspace{0.5ex}

\hspace{.8\funcindent}\begin{boxedminipage}{\funcwidth}

    \raggedright \textbf{\_\_str\_\_}(\textit{self})

    \vspace{-1.5ex}

    \rule{\textwidth}{0.5\fboxrule}
\setlength{\parskip}{2ex}
    a.\_\_str\_\_() or a

    Will be called when printing SpaCo instance a

    (section) Input:

      \begin{itemize}
      \setlength{\parskip}{0.6ex}
        \item a SpaCo class instance

      \end{itemize}

    (section) Returns:

      \begin{itemize}
      \setlength{\parskip}{0.6ex}
        \item output (string)

      \end{itemize}

    (section) Example:

\begin{alltt}
\pysrcprompt{{\textgreater}{\textgreater}{\textgreater} }y = SpaCo([[1,2,4],[1,2,2]], \pysrcstring{'GEO'}, \pysrcstring{'car'})
\pysrcprompt{{\textgreater}{\textgreater}{\textgreater} }y
\pysrcoutput{SpaCo( [[1 2 4]}
\pysrcoutput{ [1 2 2]] ), dtype=GEO,car, units=['Re', 'Re', 'Re']}\end{alltt}
    (section) Author:

      Josef Koller, Los Alamos National Lab (jkoller@lanl.gov)

    (section) Version:

      V1: 05-Mar-2010 (JK)

\setlength{\parskip}{1ex}
      Overrides: object.\_\_str\_\_

    \end{boxedminipage}

    \vspace{0.5ex}

\hspace{.8\funcindent}\begin{boxedminipage}{\funcwidth}

    \raggedright \textbf{\_\_repr\_\_}(\textit{self})

    \vspace{-1.5ex}

    \rule{\textwidth}{0.5\fboxrule}
\setlength{\parskip}{2ex}
    a.\_\_str\_\_() or a

    Will be called when printing SpaCo instance a

    (section) Input:

      \begin{itemize}
      \setlength{\parskip}{0.6ex}
        \item a SpaCo class instance

      \end{itemize}

    (section) Returns:

      \begin{itemize}
      \setlength{\parskip}{0.6ex}
        \item output (string)

      \end{itemize}

    (section) Example:

\begin{alltt}
\pysrcprompt{{\textgreater}{\textgreater}{\textgreater} }y = SpaCo([[1,2,4],[1,2,2]], \pysrcstring{'GEO'}, \pysrcstring{'car'})
\pysrcprompt{{\textgreater}{\textgreater}{\textgreater} }y
\pysrcoutput{SpaCo( [[1 2 4]}
\pysrcoutput{ [1 2 2]] ), dtype=GEO,car, units=['Re', 'Re', 'Re']}\end{alltt}
    (section) Author:

      Josef Koller, Los Alamos National Lab (jkoller@lanl.gov)

    (section) Version:

      V1: 05-Mar-2010 (JK)

\setlength{\parskip}{1ex}
      Overrides: object.\_\_repr\_\_

    \end{boxedminipage}

    \label{spacepy:spacetime:SpaCo:__getitem__}
    \index{spacepy \textit{(package)}!spacepy.spacetime \textit{(module)}!spacepy.spacetime.SpaCo \textit{(class)}!spacepy.spacetime.SpaCo.\_\_getitem\_\_ \textit{(method)}}

    \vspace{0.5ex}

\hspace{.8\funcindent}\begin{boxedminipage}{\funcwidth}

    \raggedright \textbf{\_\_getitem\_\_}(\textit{self}, \textit{idx})

    \vspace{-1.5ex}

    \rule{\textwidth}{0.5\fboxrule}
\setlength{\parskip}{2ex}
    a.\_\_getitem\_\_(idx) or a[idx]

    Will be called when requesting items in this instance

    (section) Input:

      \begin{itemize}
      \setlength{\parskip}{0.6ex}
        \item a SpaCo class instance

        \item idx (int) : interger numbers as index

      \end{itemize}

    (section) Returns:

      \begin{itemize}
      \setlength{\parskip}{0.6ex}
        \item vals (numpy array) : new values

      \end{itemize}

    (section) Example:

\begin{alltt}
\pysrcprompt{{\textgreater}{\textgreater}{\textgreater} }y = SpaCo([[1,2,4],[1,2,2]], \pysrcstring{'GEO'}, \pysrcstring{'car'})
\pysrcprompt{{\textgreater}{\textgreater}{\textgreater} }y[0] 
\pysrcoutput{array([1, 2, 4])}\end{alltt}
    (section) Author:

      Josef Koller, Los Alamos National Lab (jkoller@lanl.gov)

    (section) Version:

      V1: 05-Mar-2010 (JK) V1.1: 19-Apr-2010: now returns a SpaCo instance

\setlength{\parskip}{1ex}
    \end{boxedminipage}

    \label{spacepy:spacetime:SpaCo:__setitem__}
    \index{spacepy \textit{(package)}!spacepy.spacetime \textit{(module)}!spacepy.spacetime.SpaCo \textit{(class)}!spacepy.spacetime.SpaCo.\_\_setitem\_\_ \textit{(method)}}

    \vspace{0.5ex}

\hspace{.8\funcindent}\begin{boxedminipage}{\funcwidth}

    \raggedright \textbf{\_\_setitem\_\_}(\textit{self}, \textit{idx}, \textit{vals})

    \vspace{-1.5ex}

    \rule{\textwidth}{0.5\fboxrule}
\setlength{\parskip}{2ex}
    a.\_\_setitem\_\_(idx, vals) or a[idx] = vals

    Will be called setting items in this instance

    (section) Input:

      \begin{itemize}
      \setlength{\parskip}{0.6ex}
        \item a SpaCo class instance

        \item idx (int) : interger numbers as index

        \item vals (numpy array or list) : new values

      \end{itemize}

    (section) Example:

\begin{alltt}
\pysrcprompt{{\textgreater}{\textgreater}{\textgreater} }y = SpaCo([[1,2,4],[1,2,2]], \pysrcstring{'GEO'}, \pysrcstring{'car'})
\pysrcprompt{{\textgreater}{\textgreater}{\textgreater} }y[1] = [9,9,9]
\pysrcprompt{{\textgreater}{\textgreater}{\textgreater} }y
\pysrcoutput{SpaCo( [[1 2 4]}
\pysrcoutput{ [9 9 9]] ), dtype=GEO,car, units=['Re', 'Re', 'Re']}\end{alltt}
    (section) Author:

      Josef Koller, Los Alamos National Lab (jkoller@lanl.gov)

    (section) Version:

      V1: 05-Mar-2010 (JK)

\setlength{\parskip}{1ex}
    \end{boxedminipage}

    \label{spacepy:spacetime:SpaCo:__len__}
    \index{spacepy \textit{(package)}!spacepy.spacetime \textit{(module)}!spacepy.spacetime.SpaCo \textit{(class)}!spacepy.spacetime.SpaCo.\_\_len\_\_ \textit{(method)}}

    \vspace{0.5ex}

\hspace{.8\funcindent}\begin{boxedminipage}{\funcwidth}

    \raggedright \textbf{\_\_len\_\_}(\textit{self})

    \vspace{-1.5ex}

    \rule{\textwidth}{0.5\fboxrule}
\setlength{\parskip}{2ex}
    a.\_\_len\_\_() or len(a)

    Will be called when requesting the length, i.e. number of items

    (section) Input:

      \begin{itemize}
      \setlength{\parskip}{0.6ex}
        \item a SpaCo class instance

      \end{itemize}

    (section) Returns:

      \begin{itemize}
      \setlength{\parskip}{0.6ex}
        \item length (int number)

      \end{itemize}

    (section) Example:

\begin{alltt}
\pysrcprompt{{\textgreater}{\textgreater}{\textgreater} }y = SpaCo([[1,2,4],[1,2,2]], \pysrcstring{'GEO'}, \pysrcstring{'car'})
\pysrcprompt{{\textgreater}{\textgreater}{\textgreater} }len(y)
\pysrcoutput{2}\end{alltt}
    (section) Author:

      Josef Koller, Los Alamos National Lab (jkoller@lanl.gov)

    (section) Version:

      V1: 05-Mar-2010 (JK)

\setlength{\parskip}{1ex}
    \end{boxedminipage}

    \label{spacepy:spacetime:SpaCo:convert}
    \index{spacepy \textit{(package)}!spacepy.spacetime \textit{(module)}!spacepy.spacetime.SpaCo \textit{(class)}!spacepy.spacetime.SpaCo.convert \textit{(method)}}

    \vspace{0.5ex}

\hspace{.8\funcindent}\begin{boxedminipage}{\funcwidth}

    \raggedright \textbf{convert}(\textit{a}, \textit{returntype}, \textit{returncarsph})

    \vspace{-1.5ex}

    \rule{\textwidth}{0.5\fboxrule}
\setlength{\parskip}{2ex}
    Can be used to create a new SpaCo instance with new coordinate types

    (section) Input:

      \begin{itemize}
      \setlength{\parskip}{0.6ex}
        \item a SpaCo class instance

        \item returntype (string) : coordinate system, possible are GDZ, GEO, 
          GSM, GSE, SM, GEI MAG, SPH, RLL

        \item returncarsph (string) : coordinate type, possible 'car' for 
          cartesian and 'sph' for spherical

      \end{itemize}

    (section) Returns:

      \begin{itemize}
      \setlength{\parskip}{0.6ex}
        \item a SpaCo object

      \end{itemize}

    (section) Example:

\begin{alltt}
\pysrcprompt{{\textgreater}{\textgreater}{\textgreater} }y = SpaCo([[1,2,4],[1,2,2]], \pysrcstring{'GEO'}, \pysrcstring{'car'})
\pysrcprompt{{\textgreater}{\textgreater}{\textgreater} }y.ticktock = TickTock([\pysrcstring{'2002-02-02T12:00:00'}, \pysrcstring{'2002-02-02T12:00:00'}], \pysrcstring{'ISO'})
\pysrcprompt{{\textgreater}{\textgreater}{\textgreater} }x = y.convert(\pysrcstring{'SM'},\pysrcstring{'car'})
\pysrcprompt{{\textgreater}{\textgreater}{\textgreater} }x
\pysrcoutput{SpaCo( [[ 0.81134097  2.6493305   3.6500375 ]}
\pysrcoutput{ [ 0.92060408  2.30678864  1.68262126]] ), dtype=SM,car, units=['Re', 'Re', 'Re']}\end{alltt}
    (section) Author:

      Josef Koller, Los Alamos National Lab (jkoller@lanl.gov)

    (section) Version:

      V1: 05-Mar-2010 (JK)

\setlength{\parskip}{1ex}
    \end{boxedminipage}

    \label{spacepy:spacetime:SpaCo:__getstate__}
    \index{spacepy \textit{(package)}!spacepy.spacetime \textit{(module)}!spacepy.spacetime.SpaCo \textit{(class)}!spacepy.spacetime.SpaCo.\_\_getstate\_\_ \textit{(method)}}

    \vspace{0.5ex}

\hspace{.8\funcindent}\begin{boxedminipage}{\funcwidth}

    \raggedright \textbf{\_\_getstate\_\_}(\textit{self})

    \vspace{-1.5ex}

    \rule{\textwidth}{0.5\fboxrule}
\setlength{\parskip}{2ex}
    Is called when pickling Author: J. Koller See Also: 
    http://docs.python.org/library/pickle.html

\setlength{\parskip}{1ex}
    \end{boxedminipage}

    \label{spacepy:spacetime:SpaCo:__setstate__}
    \index{spacepy \textit{(package)}!spacepy.spacetime \textit{(module)}!spacepy.spacetime.SpaCo \textit{(class)}!spacepy.spacetime.SpaCo.\_\_setstate\_\_ \textit{(method)}}

    \vspace{0.5ex}

\hspace{.8\funcindent}\begin{boxedminipage}{\funcwidth}

    \raggedright \textbf{\_\_setstate\_\_}(\textit{self}, \textit{dict})

    \vspace{-1.5ex}

    \rule{\textwidth}{0.5\fboxrule}
\setlength{\parskip}{2ex}
    Is called when unpickling Author: J. Koller See Also: 
    http://docs.python.org/library/pickle.html

\setlength{\parskip}{1ex}
    \end{boxedminipage}


\large{\textbf{\textit{Inherited from object}}}

\begin{quote}
\_\_delattr\_\_(), \_\_format\_\_(), \_\_getattribute\_\_(), \_\_hash\_\_(), \_\_new\_\_(), \_\_reduce\_\_(), \_\_reduce\_ex\_\_(), \_\_setattr\_\_(), \_\_sizeof\_\_(), \_\_subclasshook\_\_()
\end{quote}

%%%%%%%%%%%%%%%%%%%%%%%%%%%%%%%%%%%%%%%%%%%%%%%%%%%%%%%%%%%%%%%%%%%%%%%%%%%
%%                              Properties                               %%
%%%%%%%%%%%%%%%%%%%%%%%%%%%%%%%%%%%%%%%%%%%%%%%%%%%%%%%%%%%%%%%%%%%%%%%%%%%

  \subsubsection{Properties}

    \vspace{-1cm}
\hspace{\varindent}\begin{longtable}{|p{\varnamewidth}|p{\vardescrwidth}|l}
\cline{1-2}
\cline{1-2} \centering \textbf{Name} & \centering \textbf{Description}& \\
\cline{1-2}
\endhead\cline{1-2}\multicolumn{3}{r}{\small\textit{continued on next page}}\\\endfoot\cline{1-2}
\endlastfoot\multicolumn{2}{|l|}{\textit{Inherited from object}}\\
\multicolumn{2}{|p{\varwidth}|}{\raggedright \_\_class\_\_}\\
\cline{1-2}
\end{longtable}

    \index{spacepy \textit{(package)}!spacepy.spacetime \textit{(module)}!spacepy.spacetime.SpaCo \textit{(class)}|)}

%%%%%%%%%%%%%%%%%%%%%%%%%%%%%%%%%%%%%%%%%%%%%%%%%%%%%%%%%%%%%%%%%%%%%%%%%%%
%%                           Class Description                           %%
%%%%%%%%%%%%%%%%%%%%%%%%%%%%%%%%%%%%%%%%%%%%%%%%%%%%%%%%%%%%%%%%%%%%%%%%%%%

    \index{spacepy \textit{(package)}!spacepy.spacetime \textit{(module)}!spacepy.spacetime.TickDelta \textit{(class)}|(}
\subsection{Class TickDelta}

    \label{spacepy:spacetime:TickDelta}
\begin{tabular}{cccccc}
% Line for object, linespec=[False]
\multicolumn{2}{r}{\settowidth{\BCL}{object}\multirow{2}{\BCL}{object}}
&&
  \\\cline{3-3}
  &&\multicolumn{1}{c|}{}
&&
  \\
&&\multicolumn{2}{l}{\textbf{spacepy.spacetime.TickDelta}}
\end{tabular}

TickDelta( **kwargs )

TickDelta class holding timedelta similar to datetime.timedelta This can be
used to add/substract from TickTock objects

(section) Input:

  \begin{itemize}
  \setlength{\parskip}{0.6ex}
    \item days, hours, minutes and/or seconds (int or float) : time step

  \end{itemize}

(section) Returns:

  \begin{itemize}
  \setlength{\parskip}{0.6ex}
    \item instance with self.days, self.secs, self.timedelta

  \end{itemize}

(section) Example:

\begin{alltt}
\pysrcprompt{{\textgreater}{\textgreater}{\textgreater} }dt = TickDelta(days=3.5, hours=12)
\pysrcprompt{{\textgreater}{\textgreater}{\textgreater} }dt
\pysrcoutput{TickDelta( days=4.0 )}\end{alltt}
(section) See Also:

  spacetime.TickTock class

(section) Author:

  Josef Koller, Los Alamos National Lab (jkoller@lanl.gov)

(section) Version:

  V1: 03-Mar-2010 (JK)


%%%%%%%%%%%%%%%%%%%%%%%%%%%%%%%%%%%%%%%%%%%%%%%%%%%%%%%%%%%%%%%%%%%%%%%%%%%
%%                                Methods                                %%
%%%%%%%%%%%%%%%%%%%%%%%%%%%%%%%%%%%%%%%%%%%%%%%%%%%%%%%%%%%%%%%%%%%%%%%%%%%

  \subsubsection{Methods}

    \vspace{0.5ex}

\hspace{.8\funcindent}\begin{boxedminipage}{\funcwidth}

    \raggedright \textbf{\_\_init\_\_}(**\textit{kwargs})

\setlength{\parskip}{2ex}
    x.\_\_init\_\_(...) initializes x; see x.\_\_class\_\_.\_\_doc\_\_ for 
    signature

\setlength{\parskip}{1ex}
      Overrides: object.\_\_init\_\_ 	extit{(inherited documentation)}

    \end{boxedminipage}

    \vspace{0.5ex}

\hspace{.8\funcindent}\begin{boxedminipage}{\funcwidth}

    \raggedright \textbf{\_\_str\_\_}(\textit{self})

    \vspace{-1.5ex}

    \rule{\textwidth}{0.5\fboxrule}
\setlength{\parskip}{2ex}
    dt.\_\_str\_\_() or dt

    Will be called when printing TickDelta instance dt

    (section) Input:

      \begin{itemize}
      \setlength{\parskip}{0.6ex}
        \item a TickDelta class instance

      \end{itemize}

    (section) Returns:

      \begin{itemize}
      \setlength{\parskip}{0.6ex}
        \item output (string)

      \end{itemize}

    (section) Example:

\begin{alltt}
\pysrcprompt{{\textgreater}{\textgreater}{\textgreater} }dt = TickDelta(3)
\pysrcprompt{{\textgreater}{\textgreater}{\textgreater} }dt
\pysrcoutput{TickDelta( days=3 )}\end{alltt}
    (section) Author:

      Josef Koller, Los Alamos National Lab (jkoller@lanl.gov)

    (section) Version:

      V1: 03-Mar-2010 (JK)

\setlength{\parskip}{1ex}
      Overrides: object.\_\_str\_\_

    \end{boxedminipage}

    \vspace{0.5ex}

\hspace{.8\funcindent}\begin{boxedminipage}{\funcwidth}

    \raggedright \textbf{\_\_repr\_\_}(\textit{self})

    \vspace{-1.5ex}

    \rule{\textwidth}{0.5\fboxrule}
\setlength{\parskip}{2ex}
    dt.\_\_str\_\_() or dt

    Will be called when printing TickDelta instance dt

    (section) Input:

      \begin{itemize}
      \setlength{\parskip}{0.6ex}
        \item a TickDelta class instance

      \end{itemize}

    (section) Returns:

      \begin{itemize}
      \setlength{\parskip}{0.6ex}
        \item output (string)

      \end{itemize}

    (section) Example:

\begin{alltt}
\pysrcprompt{{\textgreater}{\textgreater}{\textgreater} }dt = TickDelta(3)
\pysrcprompt{{\textgreater}{\textgreater}{\textgreater} }dt
\pysrcoutput{TickDelta( days=3 )}\end{alltt}
    (section) Author:

      Josef Koller, Los Alamos National Lab (jkoller@lanl.gov)

    (section) Version:

      V1: 03-Mar-2010 (JK)

\setlength{\parskip}{1ex}
      Overrides: object.\_\_repr\_\_

    \end{boxedminipage}

    \label{spacepy:spacetime:TickDelta:__add__}
    \index{spacepy \textit{(package)}!spacepy.spacetime \textit{(module)}!spacepy.spacetime.TickDelta \textit{(class)}!spacepy.spacetime.TickDelta.\_\_add\_\_ \textit{(method)}}

    \vspace{0.5ex}

\hspace{.8\funcindent}\begin{boxedminipage}{\funcwidth}

    \raggedright \textbf{\_\_add\_\_}(\textit{self}, \textit{other})

    \vspace{-1.5ex}

    \rule{\textwidth}{0.5\fboxrule}
\setlength{\parskip}{2ex}
    see TickTock.\_\_add\_\_

\setlength{\parskip}{1ex}
    \end{boxedminipage}

    \label{spacepy:spacetime:TickDelta:__sub__}
    \index{spacepy \textit{(package)}!spacepy.spacetime \textit{(module)}!spacepy.spacetime.TickDelta \textit{(class)}!spacepy.spacetime.TickDelta.\_\_sub\_\_ \textit{(method)}}

    \vspace{0.5ex}

\hspace{.8\funcindent}\begin{boxedminipage}{\funcwidth}

    \raggedright \textbf{\_\_sub\_\_}(\textit{self}, \textit{other})

    \vspace{-1.5ex}

    \rule{\textwidth}{0.5\fboxrule}
\setlength{\parskip}{2ex}
    see TickTock.\_\_sub\_\_

\setlength{\parskip}{1ex}
    \end{boxedminipage}

    \label{spacepy:spacetime:TickDelta:__mul__}
    \index{spacepy \textit{(package)}!spacepy.spacetime \textit{(module)}!spacepy.spacetime.TickDelta \textit{(class)}!spacepy.spacetime.TickDelta.\_\_mul\_\_ \textit{(method)}}

    \vspace{0.5ex}

\hspace{.8\funcindent}\begin{boxedminipage}{\funcwidth}

    \raggedright \textbf{\_\_mul\_\_}(\textit{self}, \textit{other})

    \vspace{-1.5ex}

    \rule{\textwidth}{0.5\fboxrule}
\setlength{\parskip}{2ex}
    see TickTock.\_\_sub\_\_

\setlength{\parskip}{1ex}
    \end{boxedminipage}

    \label{spacepy:spacetime:TickDelta:__getstate__}
    \index{spacepy \textit{(package)}!spacepy.spacetime \textit{(module)}!spacepy.spacetime.TickDelta \textit{(class)}!spacepy.spacetime.TickDelta.\_\_getstate\_\_ \textit{(method)}}

    \vspace{0.5ex}

\hspace{.8\funcindent}\begin{boxedminipage}{\funcwidth}

    \raggedright \textbf{\_\_getstate\_\_}(\textit{self})

    \vspace{-1.5ex}

    \rule{\textwidth}{0.5\fboxrule}
\setlength{\parskip}{2ex}
    Is called when pickling Author: J. Koller See Also: 
    http://docs.python.org/library/pickle.html

\setlength{\parskip}{1ex}
    \end{boxedminipage}

    \label{spacepy:spacetime:TickDelta:__setstate__}
    \index{spacepy \textit{(package)}!spacepy.spacetime \textit{(module)}!spacepy.spacetime.TickDelta \textit{(class)}!spacepy.spacetime.TickDelta.\_\_setstate\_\_ \textit{(method)}}

    \vspace{0.5ex}

\hspace{.8\funcindent}\begin{boxedminipage}{\funcwidth}

    \raggedright \textbf{\_\_setstate\_\_}(\textit{self}, \textit{dict})

    \vspace{-1.5ex}

    \rule{\textwidth}{0.5\fboxrule}
\setlength{\parskip}{2ex}
    Is called when unpickling Author: J. Koller See Also: 
    http://docs.python.org/library/pickle.html

\setlength{\parskip}{1ex}
    \end{boxedminipage}


\large{\textbf{\textit{Inherited from object}}}

\begin{quote}
\_\_delattr\_\_(), \_\_format\_\_(), \_\_getattribute\_\_(), \_\_hash\_\_(), \_\_new\_\_(), \_\_reduce\_\_(), \_\_reduce\_ex\_\_(), \_\_setattr\_\_(), \_\_sizeof\_\_(), \_\_subclasshook\_\_()
\end{quote}

%%%%%%%%%%%%%%%%%%%%%%%%%%%%%%%%%%%%%%%%%%%%%%%%%%%%%%%%%%%%%%%%%%%%%%%%%%%
%%                              Properties                               %%
%%%%%%%%%%%%%%%%%%%%%%%%%%%%%%%%%%%%%%%%%%%%%%%%%%%%%%%%%%%%%%%%%%%%%%%%%%%

  \subsubsection{Properties}

    \vspace{-1cm}
\hspace{\varindent}\begin{longtable}{|p{\varnamewidth}|p{\vardescrwidth}|l}
\cline{1-2}
\cline{1-2} \centering \textbf{Name} & \centering \textbf{Description}& \\
\cline{1-2}
\endhead\cline{1-2}\multicolumn{3}{r}{\small\textit{continued on next page}}\\\endfoot\cline{1-2}
\endlastfoot\multicolumn{2}{|l|}{\textit{Inherited from object}}\\
\multicolumn{2}{|p{\varwidth}|}{\raggedright \_\_class\_\_}\\
\cline{1-2}
\end{longtable}

    \index{spacepy \textit{(package)}!spacepy.spacetime \textit{(module)}!spacepy.spacetime.TickDelta \textit{(class)}|)}

%%%%%%%%%%%%%%%%%%%%%%%%%%%%%%%%%%%%%%%%%%%%%%%%%%%%%%%%%%%%%%%%%%%%%%%%%%%
%%                           Class Description                           %%
%%%%%%%%%%%%%%%%%%%%%%%%%%%%%%%%%%%%%%%%%%%%%%%%%%%%%%%%%%%%%%%%%%%%%%%%%%%

    \index{spacepy \textit{(package)}!spacepy.spacetime \textit{(module)}!spacepy.spacetime.TickTock \textit{(class)}|(}
\subsection{Class TickTock}

    \label{spacepy:spacetime:TickTock}
\begin{tabular}{cccccc}
% Line for object, linespec=[False]
\multicolumn{2}{r}{\settowidth{\BCL}{object}\multirow{2}{\BCL}{object}}
&&
  \\\cline{3-3}
  &&\multicolumn{1}{c|}{}
&&
  \\
&&\multicolumn{2}{l}{\textbf{spacepy.spacetime.TickTock}}
\end{tabular}

TickTock( data, dtype )

TickTock class holding various time coordinate systems (TAI, UTC, ISO, JD, 
MJD, UNX, RDT, CDF, DOY, eDOY)

Possible data types: ISO: ISO standard format like '2002-02-25T12:20:30' 
UTC: datetime object with UTC time TAI: elapsed seconds since 1958/1/1 
(includes leap seconds) UNX: elapsed seconds since 1970/1/1 (all days have 
86400 secs sometimes unequal lenghts) JD: Julian days elapsed MJD: Modified
Julian days RDT: Rata Die days elapsed since 1/1/1 CDF: CDF epoch: 
milliseconds since 1/1/0000

(section) Input:

  \begin{itemize}
  \setlength{\parskip}{0.6ex}
    \item data (int, datetime, float, string) : time stamp

    \item dtype (string) : data type for data, possible values: CDF, ISO, UTC, 
      TAI, UNX, JD, MJD, RDT

  \end{itemize}

(section) Returns:

  \begin{itemize}
  \setlength{\parskip}{0.6ex}
    \item instance with self.data, self.dtype, self.UTC etc

  \end{itemize}

(section) Example:

\begin{alltt}
\pysrcprompt{{\textgreater}{\textgreater}{\textgreater} }x=TickTock(2452331.0142361112, \pysrcstring{'JD'})
\pysrcprompt{{\textgreater}{\textgreater}{\textgreater} }x.ISO
\pysrcoutput{'2002-02-25T12:20:30'}
\pysrcoutput{}\pysrcprompt{{\textgreater}{\textgreater}{\textgreater} }x.DOY \pysrccomment{\# Day of year}
\pysrcoutput{56}\end{alltt}
(section) See Also:

  a.getCDF, a.getISO, a.getUTC, etc.

(section) Author:

  Josef Koller, Los Alamos National Lab (jkoller@lanl.gov)

(section) Version:

  V1: 20-Jan-2010 V2: 25-Jan-2010: includes array support (JK) V3: 
  25-feb-2010: pulled functions into class (JK)


%%%%%%%%%%%%%%%%%%%%%%%%%%%%%%%%%%%%%%%%%%%%%%%%%%%%%%%%%%%%%%%%%%%%%%%%%%%
%%                                Methods                                %%
%%%%%%%%%%%%%%%%%%%%%%%%%%%%%%%%%%%%%%%%%%%%%%%%%%%%%%%%%%%%%%%%%%%%%%%%%%%

  \subsubsection{Methods}

    \vspace{0.5ex}

\hspace{.8\funcindent}\begin{boxedminipage}{\funcwidth}

    \raggedright \textbf{\_\_init\_\_}(\textit{data}, \textit{dtype})

\setlength{\parskip}{2ex}
    x.\_\_init\_\_(...) initializes x; see x.\_\_class\_\_.\_\_doc\_\_ for 
    signature

\setlength{\parskip}{1ex}
      Overrides: object.\_\_init\_\_ 	extit{(inherited documentation)}

    \end{boxedminipage}

    \vspace{0.5ex}

\hspace{.8\funcindent}\begin{boxedminipage}{\funcwidth}

    \raggedright \textbf{\_\_str\_\_}(\textit{self})

    \vspace{-1.5ex}

    \rule{\textwidth}{0.5\fboxrule}
\setlength{\parskip}{2ex}
    a.\_\_str\_\_() or a

    Will be called when printing TickTock instance a

    (section) Input:

      \begin{itemize}
      \setlength{\parskip}{0.6ex}
        \item a TickTock class instance

      \end{itemize}

    (section) Returns:

      \begin{itemize}
      \setlength{\parskip}{0.6ex}
        \item output (string)

      \end{itemize}

    (section) Example:

\begin{alltt}
\pysrcprompt{{\textgreater}{\textgreater}{\textgreater} }a = TickTock(\pysrcstring{'2002-02-02T12:00:03'}, \pysrcstring{'ISO'})
\pysrcprompt{{\textgreater}{\textgreater}{\textgreater} }a
\pysrcoutput{TickTock( ['2002-02-02T12:00:00'] ), dtype=ISO}\end{alltt}
    (section) Author:

      Josef Koller, Los Alamos National Lab (jkoller@lanl.gov)

    (section) Version:

      V1: 03-Mar-2010 (JK)

\setlength{\parskip}{1ex}
      Overrides: object.\_\_str\_\_

    \end{boxedminipage}

    \vspace{0.5ex}

\hspace{.8\funcindent}\begin{boxedminipage}{\funcwidth}

    \raggedright \textbf{\_\_repr\_\_}(\textit{self})

    \vspace{-1.5ex}

    \rule{\textwidth}{0.5\fboxrule}
\setlength{\parskip}{2ex}
    a.\_\_str\_\_() or a

    Will be called when printing TickTock instance a

    (section) Input:

      \begin{itemize}
      \setlength{\parskip}{0.6ex}
        \item a TickTock class instance

      \end{itemize}

    (section) Returns:

      \begin{itemize}
      \setlength{\parskip}{0.6ex}
        \item output (string)

      \end{itemize}

    (section) Example:

\begin{alltt}
\pysrcprompt{{\textgreater}{\textgreater}{\textgreater} }a = TickTock(\pysrcstring{'2002-02-02T12:00:03'}, \pysrcstring{'ISO'})
\pysrcprompt{{\textgreater}{\textgreater}{\textgreater} }a
\pysrcoutput{TickTock( ['2002-02-02T12:00:00'] ), dtype=ISO}\end{alltt}
    (section) Author:

      Josef Koller, Los Alamos National Lab (jkoller@lanl.gov)

    (section) Version:

      V1: 03-Mar-2010 (JK)

\setlength{\parskip}{1ex}
      Overrides: object.\_\_repr\_\_

    \end{boxedminipage}

    \label{spacepy:spacetime:TickTock:__getstate__}
    \index{spacepy \textit{(package)}!spacepy.spacetime \textit{(module)}!spacepy.spacetime.TickTock \textit{(class)}!spacepy.spacetime.TickTock.\_\_getstate\_\_ \textit{(method)}}

    \vspace{0.5ex}

\hspace{.8\funcindent}\begin{boxedminipage}{\funcwidth}

    \raggedright \textbf{\_\_getstate\_\_}(\textit{self})

    \vspace{-1.5ex}

    \rule{\textwidth}{0.5\fboxrule}
\setlength{\parskip}{2ex}
    Is called when pickling Author: J. Koller See Also: 
    http://docs.python.org/library/pickle.html

\setlength{\parskip}{1ex}
    \end{boxedminipage}

    \label{spacepy:spacetime:TickTock:__setstate__}
    \index{spacepy \textit{(package)}!spacepy.spacetime \textit{(module)}!spacepy.spacetime.TickTock \textit{(class)}!spacepy.spacetime.TickTock.\_\_setstate\_\_ \textit{(method)}}

    \vspace{0.5ex}

\hspace{.8\funcindent}\begin{boxedminipage}{\funcwidth}

    \raggedright \textbf{\_\_setstate\_\_}(\textit{self}, \textit{dict})

    \vspace{-1.5ex}

    \rule{\textwidth}{0.5\fboxrule}
\setlength{\parskip}{2ex}
    Is called when unpickling Author: J. Koller See Also: 
    http://docs.python.org/library/pickle.html

\setlength{\parskip}{1ex}
    \end{boxedminipage}

    \label{spacepy:spacetime:TickTock:__getitem__}
    \index{spacepy \textit{(package)}!spacepy.spacetime \textit{(module)}!spacepy.spacetime.TickTock \textit{(class)}!spacepy.spacetime.TickTock.\_\_getitem\_\_ \textit{(method)}}

    \vspace{0.5ex}

\hspace{.8\funcindent}\begin{boxedminipage}{\funcwidth}

    \raggedright \textbf{\_\_getitem\_\_}(\textit{self}, \textit{idx})

    \vspace{-1.5ex}

    \rule{\textwidth}{0.5\fboxrule}
\setlength{\parskip}{2ex}
    a.\_\_getitem\_\_(idx) or a[idx]

    Will be called when requesting items in this instance

    (section) Input:

      \begin{itemize}
      \setlength{\parskip}{0.6ex}
        \item a TickTock class instance

        \item idx (int) : interger numbers as index

      \end{itemize}

    (section) Returns:

      \begin{itemize}
      \setlength{\parskip}{0.6ex}
        \item TickTock instance with requested values

      \end{itemize}

    (section) Example:

\begin{alltt}
\pysrcprompt{{\textgreater}{\textgreater}{\textgreater} }a = TickTock(\pysrcstring{'2002-02-02T12:00:03'}, \pysrcstring{'ISO'})
\pysrcprompt{{\textgreater}{\textgreater}{\textgreater} }a[0]
\pysrcoutput{'2002-02-02T12:00:00'}\end{alltt}
    (section) See Also:

      a.\_\_setitem\_\_

    (section) Author:

      Josef Koller, Los Alamos National Lab (jkoller@lanl.gov)

    (section) Version:

      V1.0: 03-Mar-2010 (JK) V1.1: 23-Mar-2010: now returns TickTock 
      instance and can be indexed with arrays (JK)

\setlength{\parskip}{1ex}
    \end{boxedminipage}

    \label{spacepy:spacetime:TickTock:__setitem__}
    \index{spacepy \textit{(package)}!spacepy.spacetime \textit{(module)}!spacepy.spacetime.TickTock \textit{(class)}!spacepy.spacetime.TickTock.\_\_setitem\_\_ \textit{(method)}}

    \vspace{0.5ex}

\hspace{.8\funcindent}\begin{boxedminipage}{\funcwidth}

    \raggedright \textbf{\_\_setitem\_\_}(\textit{self}, \textit{idx}, \textit{vals})

    \vspace{-1.5ex}

    \rule{\textwidth}{0.5\fboxrule}
\setlength{\parskip}{2ex}
    a.\_\_setitem\_\_(idx, vals) or a[idx] = vals

    Will be called setting items in this instance

    (section) Input:

      \begin{itemize}
      \setlength{\parskip}{0.6ex}
        \item a TickTock class instance

        \item idx (int) : interger numbers as index

        \item vals (float, string, datetime) : new values

      \end{itemize}

    (section) Example:

\begin{alltt}
\pysrcprompt{{\textgreater}{\textgreater}{\textgreater} }a = TickTock(\pysrcstring{'2002-02-02T12:00:03'}, \pysrcstring{'ISO'})
\pysrcprompt{{\textgreater}{\textgreater}{\textgreater} }a[0] = \pysrcstring{'2003-03-03T00:00:00'}\end{alltt}
    (section) See Also:

      a.\_\_getitem\_\_

    (section) Author:

      Josef Koller, Los Alamos National Lab (jkoller@lanl.gov)

    (section) Version:

      V1: 03-Mar-2010 (JK)

\setlength{\parskip}{1ex}
    \end{boxedminipage}

    \label{spacepy:spacetime:TickTock:__len__}
    \index{spacepy \textit{(package)}!spacepy.spacetime \textit{(module)}!spacepy.spacetime.TickTock \textit{(class)}!spacepy.spacetime.TickTock.\_\_len\_\_ \textit{(method)}}

    \vspace{0.5ex}

\hspace{.8\funcindent}\begin{boxedminipage}{\funcwidth}

    \raggedright \textbf{\_\_len\_\_}(\textit{self})

    \vspace{-1.5ex}

    \rule{\textwidth}{0.5\fboxrule}
\setlength{\parskip}{2ex}
    a.\_\_len\_\_() or len(a)

    Will be called when requesting the length, i.e. number of items

    (section) Input:

      \begin{itemize}
      \setlength{\parskip}{0.6ex}
        \item a TickTock class instance

      \end{itemize}

    (section) Returns:

      \begin{itemize}
      \setlength{\parskip}{0.6ex}
        \item length (int number)

      \end{itemize}

    (section) Example:

\begin{alltt}
\pysrcprompt{{\textgreater}{\textgreater}{\textgreater} }a = TickTock(\pysrcstring{'2002-02-02T12:00:03'}, \pysrcstring{'ISO'})
\pysrcprompt{{\textgreater}{\textgreater}{\textgreater} }a.len
\pysrcoutput{1}\end{alltt}
    (section) Author:

      Josef Koller, Los Alamos National Lab (jkoller@lanl.gov)

    (section) Version:

      V1: 03-Mar-2010 (JK)

\setlength{\parskip}{1ex}
    \end{boxedminipage}

    \label{spacepy:spacetime:TickTock:__cmp__}
    \index{spacepy \textit{(package)}!spacepy.spacetime \textit{(module)}!spacepy.spacetime.TickTock \textit{(class)}!spacepy.spacetime.TickTock.\_\_cmp\_\_ \textit{(method)}}

    \vspace{0.5ex}

\hspace{.8\funcindent}\begin{boxedminipage}{\funcwidth}

    \raggedright \textbf{\_\_cmp\_\_}(\textit{a}, \textit{other})

    \vspace{-1.5ex}

    \rule{\textwidth}{0.5\fboxrule}
\setlength{\parskip}{2ex}
    Will be called when two TickTock instances are compared

    (section) Input:

      \begin{itemize}
      \setlength{\parskip}{0.6ex}
        \item a TickTock class instance

        \item other (TickTock instance)

      \end{itemize}

    (section) Returns:

      \begin{itemize}
      \setlength{\parskip}{0.6ex}
        \item True or False

      \end{itemize}

    (section) Example:

\begin{alltt}
\pysrcprompt{{\textgreater}{\textgreater}{\textgreater} }a = TickTock(\pysrcstring{'2002-02-02T12:00:03'}, \pysrcstring{'ISO'})
\pysrcprompt{{\textgreater}{\textgreater}{\textgreater} }b = TickTock(\pysrcstring{'2002-02-02T12:00:00'}, \pysrcstring{'ISO'})
\pysrcprompt{{\textgreater}{\textgreater}{\textgreater} }a {\textgreater} b
\pysrcoutput{True}\end{alltt}
    (section) See Also:

      a.\_\_add\_\_, a.\_\_sub\_\_

    (section) Author:

      Josef Koller, Los Alamos National Lab (jkoller@lanl.gov)

    (section) Version:

      V1: 03-Mar-2010 (JK)

\setlength{\parskip}{1ex}
    \end{boxedminipage}

    \label{spacepy:spacetime:TickTock:__gt__}
    \index{spacepy \textit{(package)}!spacepy.spacetime \textit{(module)}!spacepy.spacetime.TickTock \textit{(class)}!spacepy.spacetime.TickTock.\_\_gt\_\_ \textit{(method)}}

    \vspace{0.5ex}

\hspace{.8\funcindent}\begin{boxedminipage}{\funcwidth}

    \raggedright \textbf{\_\_gt\_\_}(\textit{self}, \textit{other})

\setlength{\parskip}{2ex}
\setlength{\parskip}{1ex}
    \end{boxedminipage}

    \label{spacepy:spacetime:TickTock:__lt__}
    \index{spacepy \textit{(package)}!spacepy.spacetime \textit{(module)}!spacepy.spacetime.TickTock \textit{(class)}!spacepy.spacetime.TickTock.\_\_lt\_\_ \textit{(method)}}

    \vspace{0.5ex}

\hspace{.8\funcindent}\begin{boxedminipage}{\funcwidth}

    \raggedright \textbf{\_\_lt\_\_}(\textit{self}, \textit{other})

\setlength{\parskip}{2ex}
\setlength{\parskip}{1ex}
    \end{boxedminipage}

    \label{spacepy:spacetime:TickTock:__ge__}
    \index{spacepy \textit{(package)}!spacepy.spacetime \textit{(module)}!spacepy.spacetime.TickTock \textit{(class)}!spacepy.spacetime.TickTock.\_\_ge\_\_ \textit{(method)}}

    \vspace{0.5ex}

\hspace{.8\funcindent}\begin{boxedminipage}{\funcwidth}

    \raggedright \textbf{\_\_ge\_\_}(\textit{self}, \textit{other})

\setlength{\parskip}{2ex}
\setlength{\parskip}{1ex}
    \end{boxedminipage}

    \label{spacepy:spacetime:TickTock:__le__}
    \index{spacepy \textit{(package)}!spacepy.spacetime \textit{(module)}!spacepy.spacetime.TickTock \textit{(class)}!spacepy.spacetime.TickTock.\_\_le\_\_ \textit{(method)}}

    \vspace{0.5ex}

\hspace{.8\funcindent}\begin{boxedminipage}{\funcwidth}

    \raggedright \textbf{\_\_le\_\_}(\textit{self}, \textit{other})

\setlength{\parskip}{2ex}
\setlength{\parskip}{1ex}
    \end{boxedminipage}

    \label{spacepy:spacetime:TickTock:__eq__}
    \index{spacepy \textit{(package)}!spacepy.spacetime \textit{(module)}!spacepy.spacetime.TickTock \textit{(class)}!spacepy.spacetime.TickTock.\_\_eq\_\_ \textit{(method)}}

    \vspace{0.5ex}

\hspace{.8\funcindent}\begin{boxedminipage}{\funcwidth}

    \raggedright \textbf{\_\_eq\_\_}(\textit{self}, \textit{other})

\setlength{\parskip}{2ex}
\setlength{\parskip}{1ex}
    \end{boxedminipage}

    \label{spacepy:spacetime:TickTock:__ne__}
    \index{spacepy \textit{(package)}!spacepy.spacetime \textit{(module)}!spacepy.spacetime.TickTock \textit{(class)}!spacepy.spacetime.TickTock.\_\_ne\_\_ \textit{(method)}}

    \vspace{0.5ex}

\hspace{.8\funcindent}\begin{boxedminipage}{\funcwidth}

    \raggedright \textbf{\_\_ne\_\_}(\textit{self}, \textit{other})

\setlength{\parskip}{2ex}
\setlength{\parskip}{1ex}
    \end{boxedminipage}

    \label{spacepy:spacetime:TickTock:__sub__}
    \index{spacepy \textit{(package)}!spacepy.spacetime \textit{(module)}!spacepy.spacetime.TickTock \textit{(class)}!spacepy.spacetime.TickTock.\_\_sub\_\_ \textit{(method)}}

    \vspace{0.5ex}

\hspace{.8\funcindent}\begin{boxedminipage}{\funcwidth}

    \raggedright \textbf{\_\_sub\_\_}(\textit{a}, \textit{other})

    \vspace{-1.5ex}

    \rule{\textwidth}{0.5\fboxrule}
\setlength{\parskip}{2ex}
    Will be called if a TickDelta object is substracted to this instance 
    and returns a new TickTock instance

    (section) Input:

      \begin{itemize}
      \setlength{\parskip}{0.6ex}
        \item a TickTock class instance

        \item other (TickDelta instance)

      \end{itemize}

    (section) Example:

\begin{alltt}
\pysrcprompt{{\textgreater}{\textgreater}{\textgreater} }a = TickTock(\pysrcstring{'2002-02-02T12:00:00'}, \pysrcstring{'ISO'})
\pysrcprompt{{\textgreater}{\textgreater}{\textgreater} }dt = TickDelta(3)
\pysrcprompt{{\textgreater}{\textgreater}{\textgreater} }a - dt
\pysrcoutput{TickTock( ['2002-02-05T12:00:00'] ), dtype=ISO}\end{alltt}
    (section) See Also:

      \_\_sub\_\_

    (section) Author:

      Josef Koller, Los Alamos National Lab (jkoller@lanl.gov)

    (section) Version:

      V1: 03-Mar-2010 (JK)

\setlength{\parskip}{1ex}
    \end{boxedminipage}

    \label{spacepy:spacetime:TickTock:__add__}
    \index{spacepy \textit{(package)}!spacepy.spacetime \textit{(module)}!spacepy.spacetime.TickTock \textit{(class)}!spacepy.spacetime.TickTock.\_\_add\_\_ \textit{(method)}}

    \vspace{0.5ex}

\hspace{.8\funcindent}\begin{boxedminipage}{\funcwidth}

    \raggedright \textbf{\_\_add\_\_}(\textit{a}, \textit{other})

    \vspace{-1.5ex}

    \rule{\textwidth}{0.5\fboxrule}
\setlength{\parskip}{2ex}
    Will be called if a TickDelta object is added to this instance and 
    returns a new TickTock instance

    (section) Input:

      \begin{itemize}
      \setlength{\parskip}{0.6ex}
        \item a TickTock class instance

        \item other (TickDelta instance)

      \end{itemize}

    (section) Example:

\begin{alltt}
\pysrcprompt{{\textgreater}{\textgreater}{\textgreater} }a = TickTock(\pysrcstring{'2002-02-02T12:00:00'}, \pysrcstring{'ISO'})
\pysrcprompt{{\textgreater}{\textgreater}{\textgreater} }dt = TickDelta(3)
\pysrcprompt{{\textgreater}{\textgreater}{\textgreater} }a + dt
\pysrcoutput{TickTock( ['2002-02-05T12:00:00'] ), dtype=ISO}\end{alltt}
    (section) See Also:

      \_\_sub\_\_

    (section) Author:

      Josef Koller, Los Alamos National Lab (jkoller@lanl.gov)

    (section) Version:

      V1: 03-Mar-2010 (JK)

\setlength{\parskip}{1ex}
    \end{boxedminipage}

    \label{spacepy:spacetime:TickTock:__getattr__}
    \index{spacepy \textit{(package)}!spacepy.spacetime \textit{(module)}!spacepy.spacetime.TickTock \textit{(class)}!spacepy.spacetime.TickTock.\_\_getattr\_\_ \textit{(method)}}

    \vspace{0.5ex}

\hspace{.8\funcindent}\begin{boxedminipage}{\funcwidth}

    \raggedright \textbf{\_\_getattr\_\_}(\textit{a}, \textit{name})

    \vspace{-1.5ex}

    \rule{\textwidth}{0.5\fboxrule}
\setlength{\parskip}{2ex}
    Will be called if attribute "name" is not found in TickTock class 
    instance. It will add TAI, RDT, etc

    (section) Input:

      \begin{itemize}
      \setlength{\parskip}{0.6ex}
        \item a TickTock class instance

        \item name (string) : a string from the list of time systems 'UTC', 
          'TAI', 'ISO', 'JD', 'MJD', 'UNX', 'RDT', 'CDF', 'DOY', 'eDOY', 
          'leaps'

      \end{itemize}

    (section) Returns:

      \begin{itemize}
      \setlength{\parskip}{0.6ex}
        \item requested values as either list/numpy array

      \end{itemize}

    (section) Author:

      Josef Koller, Los Alamos National Lab (jkoller@lanl.gov)

    (section) Version:

      V1: 03-Mar-2010 (JK)

\setlength{\parskip}{1ex}
    \end{boxedminipage}

    \label{spacepy:spacetime:TickTock:update_items}
    \index{spacepy \textit{(package)}!spacepy.spacetime \textit{(module)}!spacepy.spacetime.TickTock \textit{(class)}!spacepy.spacetime.TickTock.update\_items \textit{(method)}}

    \vspace{0.5ex}

\hspace{.8\funcindent}\begin{boxedminipage}{\funcwidth}

    \raggedright \textbf{update\_items}(\textit{a}, \textit{b}, \textit{attrib})

    \vspace{-1.5ex}

    \rule{\textwidth}{0.5\fboxrule}
\setlength{\parskip}{2ex}
    After changing the self.data attribute by either \_\_setitem\_\_ or 
    \_\_add\_\_ etc this function will update all other attributes. This 
    function is called automatically in \_\_add\_\_ and \_\_setitem\_\_

    (section) Input:

      \begin{itemize}
      \setlength{\parskip}{0.6ex}
        \item a TickTock class instance

      \end{itemize}

    (section) See Also:

      \_\_setitem\_\_, \_\_add\_\_, \_\_sub\_\_

    (section) Author:

      Josef Koller, Los Alamos National Lab (jkoller@lanl.gov)

    (section) Version:

      V1: 03-Mar-2010 (JK)

\setlength{\parskip}{1ex}
    \end{boxedminipage}

    \label{spacepy:spacetime:TickTock:convert}
    \index{spacepy \textit{(package)}!spacepy.spacetime \textit{(module)}!spacepy.spacetime.TickTock \textit{(class)}!spacepy.spacetime.TickTock.convert \textit{(method)}}

    \vspace{0.5ex}

\hspace{.8\funcindent}\begin{boxedminipage}{\funcwidth}

    \raggedright \textbf{convert}(\textit{a}, \textit{dtype})

    \vspace{-1.5ex}

    \rule{\textwidth}{0.5\fboxrule}
\setlength{\parskip}{2ex}
    convert a TickTock instance into a new time coordinate system provided 
    in dtype

    (section) Input:

      \begin{itemize}
      \setlength{\parskip}{0.6ex}
        \item a TickTock class instance

        \item dtype (string) : data type for new system, possible values are 
          CDF, ISO, UTC, TAI, UNX, JD, MJD, RDT

      \end{itemize}

    (section) Returns:

      \begin{itemize}
      \setlength{\parskip}{0.6ex}
        \item newobj (TickTock instance) with new time coordinates

      \end{itemize}

    (section) Example:

\begin{alltt}
\pysrcprompt{{\textgreater}{\textgreater}{\textgreater} }a = TickTock([\pysrcstring{'2002-02-02T12:00:00'}, \pysrcstring{'2002-02-02T12:00:00'}], \pysrcstring{'ISO'})
\pysrcprompt{{\textgreater}{\textgreater}{\textgreater} }s = a.convert(\pysrcstring{'TAI'})
\pysrcprompt{{\textgreater}{\textgreater}{\textgreater} }type(s)
\pysrcoutput{{\textless}class 'spacetime.TickTock'{\textgreater}}
\pysrcoutput{}\pysrcprompt{{\textgreater}{\textgreater}{\textgreater} }s
\pysrcoutput{TickTock( [1391342432 1391342432] ), dtype=TAI}\end{alltt}
    (section) See Also:

      a.CDF, a.ISO, a.UTC, etc.

    (section) Author:

      Josef Koller, Los Alamos National Lab (jkoller@lanl.gov)

    (section) Version:

      V1: 05-Mar-2010 (JK)

\setlength{\parskip}{1ex}
    \end{boxedminipage}

    \label{spacepy:spacetime:TickTock:append}
    \index{spacepy \textit{(package)}!spacepy.spacetime \textit{(module)}!spacepy.spacetime.TickTock \textit{(class)}!spacepy.spacetime.TickTock.append \textit{(method)}}

    \vspace{0.5ex}

\hspace{.8\funcindent}\begin{boxedminipage}{\funcwidth}

    \raggedright \textbf{append}(\textit{a}, \textit{other})

    \vspace{-1.5ex}

    \rule{\textwidth}{0.5\fboxrule}
\setlength{\parskip}{2ex}
    Will be called when another TickTock instance has to be appended to the
    current one

    (section) Input:

      \begin{itemize}
      \setlength{\parskip}{0.6ex}
        \item a TickTock class instance

        \item other (TickTock instance)

      \end{itemize}

    (section) Example:

    (section) Author:

      Josef Koller, Los Alamos National Lab (jkoller@lanl.gov)

    (section) Version:

      V1: 23-Mar-2010 (JK)

\setlength{\parskip}{1ex}
    \end{boxedminipage}

    \label{spacepy:spacetime:TickTock:getCDF}
    \index{spacepy \textit{(package)}!spacepy.spacetime \textit{(module)}!spacepy.spacetime.TickTock \textit{(class)}!spacepy.spacetime.TickTock.getCDF \textit{(method)}}

    \vspace{0.5ex}

\hspace{.8\funcindent}\begin{boxedminipage}{\funcwidth}

    \raggedright \textbf{getCDF}(\textit{self})

    \vspace{-1.5ex}

    \rule{\textwidth}{0.5\fboxrule}
\setlength{\parskip}{2ex}
    a.getCDF() or a.CDF

    Return CDF time which is milliseconds since 01-Jan-0000 00:00:00.000. 
    Year zero" is a convention chosen by NSSDC to measure epoch values. 
    This date is more commonly referred to as 1 BC. Remember that 1 BC was 
    a leap year. The CDF date/time calculations do not take into account 
    the changes to the Gregorian calendar, and cannot be directly converted
    into Julian date/times.

    (section) Input:

      \begin{itemize}
      \setlength{\parskip}{0.6ex}
        \item a TickTock class instance

      \end{itemize}

    (section) Returns:

      \begin{itemize}
      \setlength{\parskip}{0.6ex}
        \item CDF (numpy array) : days elapsed since Jan. 1st

      \end{itemize}

    (section) Example:

\begin{alltt}
\pysrcprompt{{\textgreater}{\textgreater}{\textgreater} }a = TickTock(\pysrcstring{'2002-02-02T12:00:00'}, \pysrcstring{'ISO'})
\pysrcprompt{{\textgreater}{\textgreater}{\textgreater} }a.CDF
\pysrcoutput{array([  6.31798704e+13])}\end{alltt}
    (section) See Also:

      getUTC, getUNX, getRDT, getJD, getMJD, getISO, getTAI, getDOY, 
      geteDOY

    (section) Author:

      Josef Koller, Los Alamos National Lab (jkoller@lanl.gov)

    (section) Version:

      V1: 02-Feb-2010 (JK)

\setlength{\parskip}{1ex}
    \end{boxedminipage}

    \label{spacepy:spacetime:TickTock:getDOY}
    \index{spacepy \textit{(package)}!spacepy.spacetime \textit{(module)}!spacepy.spacetime.TickTock \textit{(class)}!spacepy.spacetime.TickTock.getDOY \textit{(method)}}

    \vspace{0.5ex}

\hspace{.8\funcindent}\begin{boxedminipage}{\funcwidth}

    \raggedright \textbf{getDOY}(\textit{self})

    \vspace{-1.5ex}

    \rule{\textwidth}{0.5\fboxrule}
\setlength{\parskip}{2ex}
    a.DOY or a.getDOY()

    extract DOY (days since January 1st of given year)

    (section) Input:

      \begin{itemize}
      \setlength{\parskip}{0.6ex}
        \item a TickTock class instance

      \end{itemize}

    (section) Returns:

      \begin{itemize}
      \setlength{\parskip}{0.6ex}
        \item DOY (numpy array int) : day of the year

      \end{itemize}

    (section) Example:

\begin{alltt}
\pysrcprompt{{\textgreater}{\textgreater}{\textgreater} }a = TickTock(\pysrcstring{'2002-02-02T12:00:00'}, \pysrcstring{'ISO'})
\pysrcprompt{{\textgreater}{\textgreater}{\textgreater} }a.DOY
\pysrcoutput{array([ 33])}\end{alltt}
    (section) See Also:

      getUTC, getUNX, getRDT, getJD, getMJD, getCDF, getTAI, getISO, 
      geteDOY

    (section) Author:

      Josef Koller, Los Alamos National Lab (jkoller@lanl.gov)

    (section) Version:

      V1: 25-Jan-2010 (JK) V1.1: 20-Apr-2010: returns true DOY per 
      definition as integer (JK)

\setlength{\parskip}{1ex}
    \end{boxedminipage}

    \label{spacepy:spacetime:TickTock:geteDOY}
    \index{spacepy \textit{(package)}!spacepy.spacetime \textit{(module)}!spacepy.spacetime.TickTock \textit{(class)}!spacepy.spacetime.TickTock.geteDOY \textit{(method)}}

    \vspace{0.5ex}

\hspace{.8\funcindent}\begin{boxedminipage}{\funcwidth}

    \raggedright \textbf{geteDOY}(\textit{self})

    \vspace{-1.5ex}

    \rule{\textwidth}{0.5\fboxrule}
\setlength{\parskip}{2ex}
    a.eDOY or a.geteDOY()

    extract eDOY (elapsed days since midnight January 1st of given year)

    (section) Input:

      \begin{itemize}
      \setlength{\parskip}{0.6ex}
        \item a TickTock class instance

      \end{itemize}

    (section) Returns:

      \begin{itemize}
      \setlength{\parskip}{0.6ex}
        \item eDOY (numpy array) : days elapsed since midnight bbedJan. 1st

      \end{itemize}

    (section) Example:

\begin{alltt}
\pysrcprompt{{\textgreater}{\textgreater}{\textgreater} }a = TickTock(\pysrcstring{'2002-02-02T12:00:00'}, \pysrcstring{'ISO'})
\pysrcprompt{{\textgreater}{\textgreater}{\textgreater} }a.eDOY
\pysrcoutput{array([ 32.5])}\end{alltt}
    (section) See Also:

      getUTC, getUNX, getRDT, getJD, getMJD, getCDF, getTAI, getISO, getDOY

    (section) Author:

      Josef Koller, Los Alamos National Lab (jkoller@lanl.gov)

    (section) Version:

      V1: 20-Apr-2010 (JK)

\setlength{\parskip}{1ex}
    \end{boxedminipage}

    \label{spacepy:spacetime:TickTock:getJD}
    \index{spacepy \textit{(package)}!spacepy.spacetime \textit{(module)}!spacepy.spacetime.TickTock \textit{(class)}!spacepy.spacetime.TickTock.getJD \textit{(method)}}

    \vspace{0.5ex}

\hspace{.8\funcindent}\begin{boxedminipage}{\funcwidth}

    \raggedright \textbf{getJD}(\textit{self})

    \vspace{-1.5ex}

    \rule{\textwidth}{0.5\fboxrule}
\setlength{\parskip}{2ex}
    a.JD or a.getJD()

    convert dtype data into Julian Date (JD)

    (section) Input:

      \begin{itemize}
      \setlength{\parskip}{0.6ex}
        \item a TickTock class instance

      \end{itemize}

    (section) Returns:

      \begin{itemize}
      \setlength{\parskip}{0.6ex}
        \item JD (numpy array) : elapsed days since 12:00 January 1, 4713 BC

      \end{itemize}

    (section) Example:

\begin{alltt}
\pysrcprompt{{\textgreater}{\textgreater}{\textgreater} }a = TickTock(\pysrcstring{'2002-02-02T12:00:00'}, \pysrcstring{'ISO'})
\pysrcprompt{{\textgreater}{\textgreater}{\textgreater} }a.JD
\pysrcoutput{array([ 2452308.])}\end{alltt}
    (section) See Also:

      getUTC, getUNX, getRDT, getISO, getMJD, getCDF, getTAI, getDOY, 
      geteDOY

    (section) Author:

      Josef Koller, Los Alamos National Lab (jkoller@lanl.gov)

    (section) Version:

      V1: 20-Jan-2010 (JK) V2: 25-Jan-2010: added array support (JK)

\setlength{\parskip}{1ex}
    \end{boxedminipage}

    \label{spacepy:spacetime:TickTock:getMJD}
    \index{spacepy \textit{(package)}!spacepy.spacetime \textit{(module)}!spacepy.spacetime.TickTock \textit{(class)}!spacepy.spacetime.TickTock.getMJD \textit{(method)}}

    \vspace{0.5ex}

\hspace{.8\funcindent}\begin{boxedminipage}{\funcwidth}

    \raggedright \textbf{getMJD}(\textit{self})

    \vspace{-1.5ex}

    \rule{\textwidth}{0.5\fboxrule}
\setlength{\parskip}{2ex}
    a.MJD or a.getMJD()

    convert dtype data into MJD (modified Julian date)

    (section) Input:

      \begin{itemize}
      \setlength{\parskip}{0.6ex}
        \item a TickTock class instance

      \end{itemize}

    (section) Returns:

      \begin{itemize}
      \setlength{\parskip}{0.6ex}
        \item MJD (numpy array) : elapsed days since November 17, 1858 (Julian 
          date was 2,400 000)

      \end{itemize}

    (section) Example:

\begin{alltt}
\pysrcprompt{{\textgreater}{\textgreater}{\textgreater} }a = TickTock(\pysrcstring{'2002-02-02T12:00:00'}, \pysrcstring{'ISO'})
\pysrcprompt{{\textgreater}{\textgreater}{\textgreater} }a.MJD
\pysrcoutput{array([ 52307.5])}\end{alltt}
    (section) See Also:

      getUTC, getUNX, getRDT, getJD, getISO, getCDF, getTAI, getDOY, 
      geteDOY

    (section) Author:

      Josef Koller, Los Alamos National Lab (jkoller@lanl.gov)

    (section) Version:

      V1: 20-Jan-2010 (JK) V2: 25-Jan-2010: added support for arrays (JK)

\setlength{\parskip}{1ex}
    \end{boxedminipage}

    \label{spacepy:spacetime:TickTock:getUNX}
    \index{spacepy \textit{(package)}!spacepy.spacetime \textit{(module)}!spacepy.spacetime.TickTock \textit{(class)}!spacepy.spacetime.TickTock.getUNX \textit{(method)}}

    \vspace{0.5ex}

\hspace{.8\funcindent}\begin{boxedminipage}{\funcwidth}

    \raggedright \textbf{getUNX}(\textit{self})

    \vspace{-1.5ex}

    \rule{\textwidth}{0.5\fboxrule}
\setlength{\parskip}{2ex}
    a.UNX or a.getUNX()

    convert dtype data into Unix Time (Posix Time) seconds since 1970-Jan-1
    (not counting leap seconds)

    (section) Input:

      \begin{itemize}
      \setlength{\parskip}{0.6ex}
        \item a TickTock class instance

      \end{itemize}

    (section) Returns:

      \begin{itemize}
      \setlength{\parskip}{0.6ex}
        \item UNX (numpy array) : elapsed secs since 1970/1/1 (not counting 
          leap secs)

      \end{itemize}

    (section) Example:

\begin{alltt}
\pysrcprompt{{\textgreater}{\textgreater}{\textgreater} }a = TickTock(\pysrcstring{'2002-02-02T12:00:00'}, \pysrcstring{'ISO'})
\pysrcprompt{{\textgreater}{\textgreater}{\textgreater} }a.UNX
\pysrcoutput{array([  1.01265120e+09])}\end{alltt}
    (section) See Also:

      getUTC, getISO, getRDT, getJD, getMJD, getCDF, getTAI, getDOY, 
      geteDOY

    (section) Author:

      Josef Koller, Los Alamos National Lab (jkoller@lanl.gov)

    (section) Version:

      V1: 20-Jan-2010 (JK) V2: 25-Jan-2010: added array support (JK)

\setlength{\parskip}{1ex}
    \end{boxedminipage}

    \label{spacepy:spacetime:TickTock:getRDT}
    \index{spacepy \textit{(package)}!spacepy.spacetime \textit{(module)}!spacepy.spacetime.TickTock \textit{(class)}!spacepy.spacetime.TickTock.getRDT \textit{(method)}}

    \vspace{0.5ex}

\hspace{.8\funcindent}\begin{boxedminipage}{\funcwidth}

    \raggedright \textbf{getRDT}(\textit{self})

    \vspace{-1.5ex}

    \rule{\textwidth}{0.5\fboxrule}
\setlength{\parskip}{2ex}
    a.RDT or a.RDT()

    convert dtype data into Rata Die (lat.) Time (days since 1/1/0001)

    (section) Input:

      \begin{itemize}
      \setlength{\parskip}{0.6ex}
        \item a TickTock class instance

      \end{itemize}

    (section) Returns:

      \begin{itemize}
      \setlength{\parskip}{0.6ex}
        \item RDT (numpy array) : elapsed days since 1/1/1

      \end{itemize}

    (section) Example:

\begin{alltt}
\pysrcprompt{{\textgreater}{\textgreater}{\textgreater} }a = TickTock(\pysrcstring{'2002-02-02T12:00:00'}, \pysrcstring{'ISO'})
\pysrcprompt{{\textgreater}{\textgreater}{\textgreater} }a.RDT
\pysrcoutput{array([ 730883.5])}\end{alltt}
    (section) See Also:

      getUTC, getUNX, getISO, getJD, getMJD, getCDF, getTAI, getDOY, 
      geteDOY

    (section) Author:

      Josef Koller, Los Alamos National Lab (jkoller@lanl.gov)

    (section) Version:

      V1: 20-Jan-2010 (JK) V2: 25-Jan-2010: added array support (JK)

\setlength{\parskip}{1ex}
    \end{boxedminipage}

    \label{spacepy:spacetime:TickTock:getUTC}
    \index{spacepy \textit{(package)}!spacepy.spacetime \textit{(module)}!spacepy.spacetime.TickTock \textit{(class)}!spacepy.spacetime.TickTock.getUTC \textit{(method)}}

    \vspace{0.5ex}

\hspace{.8\funcindent}\begin{boxedminipage}{\funcwidth}

    \raggedright \textbf{getUTC}(\textit{self})

    \vspace{-1.5ex}

    \rule{\textwidth}{0.5\fboxrule}
\setlength{\parskip}{2ex}
    a.UTC or a.getUTC()

    convert dtype data into UTC object a la datetime()

    (section) Input:

      \begin{itemize}
      \setlength{\parskip}{0.6ex}
        \item a TickTock class instance

      \end{itemize}

    (section) Returns:

      \begin{itemize}
      \setlength{\parskip}{0.6ex}
        \item UTC (list of datetime objects) : datetime object in UTC time

      \end{itemize}

    (section) Example:

\begin{alltt}
\pysrcprompt{{\textgreater}{\textgreater}{\textgreater} }a = TickTock(\pysrcstring{'2002-02-02T12:00:00'}, \pysrcstring{'ISO'})
\pysrcprompt{{\textgreater}{\textgreater}{\textgreater} }a.UTC
\pysrcoutput{[datetime.datetime(2002, 2, 2, 12, 0)]}\end{alltt}
    (section) See Also:

      getISO, getUNX, getRDT, getJD, getMJD, getCDF, getTAI, getDOY, 
      geteDOY

    (section) Author:

      Josef Koller, Los Alamos National Lab (jkoller@lanl.gov)

    (section) Version:

      V1: 20-Jan-2010 (JK) V2: 25-Jan-2010: added array support (JK)

\setlength{\parskip}{1ex}
    \end{boxedminipage}

    \label{spacepy:spacetime:TickTock:getGPS}
    \index{spacepy \textit{(package)}!spacepy.spacetime \textit{(module)}!spacepy.spacetime.TickTock \textit{(class)}!spacepy.spacetime.TickTock.getGPS \textit{(method)}}

    \vspace{0.5ex}

\hspace{.8\funcindent}\begin{boxedminipage}{\funcwidth}

    \raggedright \textbf{getGPS}(\textit{self})

    \vspace{-1.5ex}

    \rule{\textwidth}{0.5\fboxrule}
\setlength{\parskip}{2ex}
    a.GPS or a.getGPS()

    return GPS epoch (0000 UT (midnight) on January 6, 1980)

    (section) Input:

      \begin{itemize}
      \setlength{\parskip}{0.6ex}
        \item a TickTock class instance

      \end{itemize}

    (section) Returns:

      \begin{itemize}
      \setlength{\parskip}{0.6ex}
        \item GPS (numpy array) : elapsed secs since 6Jan1980 (excludes leap 
          secs)

      \end{itemize}

    (section) Example:

\begin{alltt}
\pysrcprompt{{\textgreater}{\textgreater}{\textgreater} }a = TickTock(\pysrcstring{'2002-02-02T12:00:00'}, \pysrcstring{'ISO'})
\pysrcprompt{{\textgreater}{\textgreater}{\textgreater} }a.GPS
\pysrcoutput{array([])}\end{alltt}
    (section) See Also:

      getUTC, getUNX, getRDT, getJD, getMJD, getCDF, getISO, getDOY, 
      geteDOY

    (section) Author:

      Brian Larsen, Los Alamos National Lab (balarsen@lanl.gov)

    (section) Version:

      V1: 20-Jan-2010 (BAL)

\setlength{\parskip}{1ex}
    \end{boxedminipage}

    \label{spacepy:spacetime:TickTock:getTAI}
    \index{spacepy \textit{(package)}!spacepy.spacetime \textit{(module)}!spacepy.spacetime.TickTock \textit{(class)}!spacepy.spacetime.TickTock.getTAI \textit{(method)}}

    \vspace{0.5ex}

\hspace{.8\funcindent}\begin{boxedminipage}{\funcwidth}

    \raggedright \textbf{getTAI}(\textit{self})

    \vspace{-1.5ex}

    \rule{\textwidth}{0.5\fboxrule}
\setlength{\parskip}{2ex}
    a.TAI or a.getTAI()

    return TAI (International Atomic Time)

    (section) Input:

      \begin{itemize}
      \setlength{\parskip}{0.6ex}
        \item a TickTock class instance

      \end{itemize}

    (section) Returns:

      \begin{itemize}
      \setlength{\parskip}{0.6ex}
        \item TAI (numpy array) : elapsed secs since 1958/1/1 (includes leap 
          secs, i.e. all secs have equal lengths)

      \end{itemize}

    (section) Example:

\begin{alltt}
\pysrcprompt{{\textgreater}{\textgreater}{\textgreater} }a = TickTock(\pysrcstring{'2002-02-02T12:00:00'}, \pysrcstring{'ISO'})
\pysrcprompt{{\textgreater}{\textgreater}{\textgreater} }a.TAI
\pysrcoutput{array([1391342432])}\end{alltt}
    (section) See Also:

      getUTC, getUNX, getRDT, getJD, getMJD, getCDF, getISO, getDOY, 
      geteDOY

    (section) Author:

      Josef Koller, Los Alamos National Lab (jkoller@lanl.gov)

    (section) Version:

      V1: 20-Jan-2010 (JK) V2: 25-Jan-2010: include array support (JK)

\setlength{\parskip}{1ex}
    \end{boxedminipage}

    \label{spacepy:spacetime:TickTock:getISO}
    \index{spacepy \textit{(package)}!spacepy.spacetime \textit{(module)}!spacepy.spacetime.TickTock \textit{(class)}!spacepy.spacetime.TickTock.getISO \textit{(method)}}

    \vspace{0.5ex}

\hspace{.8\funcindent}\begin{boxedminipage}{\funcwidth}

    \raggedright \textbf{getISO}(\textit{self})

    \vspace{-1.5ex}

    \rule{\textwidth}{0.5\fboxrule}
\setlength{\parskip}{2ex}
    a.ISO or a.getISO()

    convert dtype data into ISO string

    (section) Input:

      \begin{itemize}
      \setlength{\parskip}{0.6ex}
        \item a TickTock class instance

      \end{itemize}

    (section) Returns:

      \begin{itemize}
      \setlength{\parskip}{0.6ex}
        \item ISO (list of strings) : date in ISO format

      \end{itemize}

    (section) Example:

\begin{alltt}
\pysrcprompt{{\textgreater}{\textgreater}{\textgreater} }a = TickTock(\pysrcstring{'2002-02-02T12:00:00'}, \pysrcstring{'ISO'})
\pysrcprompt{{\textgreater}{\textgreater}{\textgreater} }a.ISO
\pysrcoutput{['2002-02-02T12:00:00']}\end{alltt}
    (section) See Also:

      getUTC, getUNX, getRDT, getJD, getMJD, getCDF, getTAI, getDOY, 
      geteDOY

    (section) Author:

      Josef Koller, Los Alamos National Lab (jkoller@lanl.gov)

    (section) Version:

      V1: 20-Jan-2010 (JK) V2: 25-Jan-2010: included arary support (JK) V3:
      10-May-2010: speedup, arange to xrange (BAL)

\setlength{\parskip}{1ex}
    \end{boxedminipage}

    \label{spacepy:spacetime:TickTock:getleapsecs}
    \index{spacepy \textit{(package)}!spacepy.spacetime \textit{(module)}!spacepy.spacetime.TickTock \textit{(class)}!spacepy.spacetime.TickTock.getleapsecs \textit{(method)}}

    \vspace{0.5ex}

\hspace{.8\funcindent}\begin{boxedminipage}{\funcwidth}

    \raggedright \textbf{getleapsecs}(\textit{self})

    \vspace{-1.5ex}

    \rule{\textwidth}{0.5\fboxrule}
\setlength{\parskip}{2ex}
    a.leaps or a.getleapsecs()

    retrieve leapseconds from lookup table, used in getTAI

    (section) Input:

      \begin{itemize}
      \setlength{\parskip}{0.6ex}
        \item a TickTock class instance

      \end{itemize}

    (section) Returns:

      \begin{itemize}
      \setlength{\parskip}{0.6ex}
        \item secs (numpy array) : leap seconds

      \end{itemize}

    (section) Example:

\begin{alltt}
\pysrcprompt{{\textgreater}{\textgreater}{\textgreater} }a = TickTock(\pysrcstring{'2002-02-02T12:00:00'}, \pysrcstring{'ISO'})
\pysrcprompt{{\textgreater}{\textgreater}{\textgreater} }a.leaps
\pysrcoutput{array([32])}\end{alltt}
    (section) See Also:

      getTAI

    (section) Author:

      Josef Koller, Los Alamos National Lab (jkoller@lanl.gov)

    (section) Version:

      V1: 20-Jan-2010: includes array support (JK)

\setlength{\parskip}{1ex}
    \end{boxedminipage}


\large{\textbf{\textit{Inherited from object}}}

\begin{quote}
\_\_delattr\_\_(), \_\_format\_\_(), \_\_getattribute\_\_(), \_\_hash\_\_(), \_\_new\_\_(), \_\_reduce\_\_(), \_\_reduce\_ex\_\_(), \_\_setattr\_\_(), \_\_sizeof\_\_(), \_\_subclasshook\_\_()
\end{quote}

%%%%%%%%%%%%%%%%%%%%%%%%%%%%%%%%%%%%%%%%%%%%%%%%%%%%%%%%%%%%%%%%%%%%%%%%%%%
%%                              Properties                               %%
%%%%%%%%%%%%%%%%%%%%%%%%%%%%%%%%%%%%%%%%%%%%%%%%%%%%%%%%%%%%%%%%%%%%%%%%%%%

  \subsubsection{Properties}

    \vspace{-1cm}
\hspace{\varindent}\begin{longtable}{|p{\varnamewidth}|p{\vardescrwidth}|l}
\cline{1-2}
\cline{1-2} \centering \textbf{Name} & \centering \textbf{Description}& \\
\cline{1-2}
\endhead\cline{1-2}\multicolumn{3}{r}{\small\textit{continued on next page}}\\\endfoot\cline{1-2}
\endlastfoot\multicolumn{2}{|l|}{\textit{Inherited from object}}\\
\multicolumn{2}{|p{\varwidth}|}{\raggedright \_\_class\_\_}\\
\cline{1-2}
\end{longtable}

    \index{spacepy \textit{(package)}!spacepy.spacetime \textit{(module)}!spacepy.spacetime.TickTock \textit{(class)}|)}
    \index{spacepy \textit{(package)}!spacepy.spacetime \textit{(module)}|)}
