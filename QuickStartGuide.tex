\documentclass[11pt]{amsart}
\usepackage{geometry}                % See geometry.pdf to learn the layout options. There are lots.
\geometry{letterpaper}                   % ... or a4paper or a5paper or ... 
%\geometry{landscape}                % Activate for for rotated page geometry
\usepackage[parfill]{parskip}    % Activate to begin paragraphs with an empty line rather than an indent
\usepackage{graphicx}
\usepackage{amssymb}
\usepackage{epstopdf}
\DeclareGraphicsRule{.tif}{png}{.png}{`convert #1 `dirname #1`/`basename #1 .tif`.png}

\title{SpacePy - Quickstart}
\author{Josef Koller and others ...}
\date{}                                           % Activate to display a given date or no date

\begin{document}
\maketitle

%%%%%%%%%%%%%%%%%%%%%%%%
\section{Installation}

Option 1) to install it in the standard loction (depending on your system)

\texttt{>>> python setup.py install}

Option 2) to install in custom location, e.g.

\texttt{>>> python setup.py install --home=/n/packages/lib/python}

%%%%%%%%%%%%%%%%%%%%%%%%
\section{Toolbox - a box full of tools}

Contains tools that don't fit anywhere else but are, in general, quite useful. The following function are implemented....

%%%%%%%%%%%%%%%%%%%%%%%%
\section{Time and Coordinate Transformations}

Import the module as \texttt{import spacepy.spacetime as st}

%%%%%%%%%%%%%%%%%%%%%%%%
\subsection{TickTock class}

The TickTock class provides as number of time conversion routines. The following time coordinates are provided

\begin{description}
\item[UTC] Coordinated Universal Time implemented as a \texttt{datetime.datetime} class
\item[ISO] standard ISO 8601 format like \texttt{2002-10-25T14:33:59}
\item[TAI] International Atomic Time in units of seconds since Jan 1, 1958 (midnight) and includes leap seconds, i.e. every second has the same length
\item[JD] Julian Day
\item[MJD] Modified Julian Day
\item[UNX] UNIX time in seconds since Jan 1, 1970
\item[RDT] Rata Die Time (lat. fixed date) in days since Jan 1, 1 AD midnight
\item[CDF] CDF Epoch time in milliseconds since Jan 1, 0 
\item[DOY] Day of Year including fractions
\item[leaps] Leap seconds according to ftp://maia.usno.navy.mil/ser7/tai-utc.dat 
\end{description}

To access these time coordinates, you'll create an instance of a TickTock class, e.g. 

\texttt{>>> t = st.TickTock('2002-10-25T12:30:00', 'ISO')}. 

Instead of ISO you may use any of the formats listed above. You can also use numpy arrays or lists of time points. \texttt{t} has now the class attributes \texttt{t.dtype = 'ISO', t.data = '2002-10-25T12:30:00', t.UTC = datetime.datetime(2002, 10, 25, 12, 30)}. UTC is added automatically.

If you want to convert/add a class attribute from the list above, simply type e.g.
\texttt{t.RTD} or \texttt{t.getRTD()}. You can replace RTD with any from the list above.

You can find out how many leap seconds were used by issuing the command

\texttt{>>> t.getleapsecs()}


%%%%%%%%%%%%%%%%%%%%%%%%
\subsection{TickDelta class}

You can add/substract time from a TickTock class instance by creating a TickDelta instance first. 

\texttt{>>> dt = st.TickDelta(days=2.3)}

Then you can add by e.g. \texttt{t+dt} 

%%%%%%%%%%%%%%%%%%%%%%%%
\subsection{SpaCo class}

The spatial coordinate class includes the following coordinate systems in cartesian and sphericals. 

\begin{description}
\item[GDZ] (altitude, latitude, longitude in km, deg, deg
\item[GEO] cartesian, Re
\item[GSM] cartesian, Re
\item[GSE] cartesian, Re
\item[SM] cartesian, Re
\item[GEI] cartesian, Re
\item[MAG] cartesian, Re
\item[SPH] same as GEO but in spherical
\item[RLL] radial distance, latitude, longitude, Re, deg, deg.
\end{description}

 Create a SpaCo instance with spherical='sph' or cartesian='car' coordinates:
 
 \texttt{>>> coord = st.SpaCo([[1,2,4],[1,2,2]], 'GEO', 'car')}
 
 This will let you request for example all y-coordinates by \texttt{coord.y} or if given in spherical coordinates by \texttt{coord.lati}. One can transform the coordinates by \texttt{newcoord = coord.convert('GSM', 'sph')}. This will return GSM coordinates in a spherical system. Since GSM coordinates depend on time, you'll have to add first a TickTock vector like \texttt{coord.ticktock = st.TickTock(['2002-02-02T12:00:00', '2002-02-02T12:00:00'], 'ISO')}
 
  %%%%%%%%%%%%%%%%%%%%%%%%
\section{RadBelt Module}

The radiation belt module currently include a simple radial diffusion code as a class. Import the module and create a class

\texttt{>>> import spacepy.radbelt as sprb}

\texttt{>>> rb = sprb.RBmodel()}

Add a time grid for a particular period that you are interested in:

\texttt{>>> rb.setup\_ticks('2002-02-01T00:00:00', '2002-02-10T00:00:00', 0.25)}

This will automatically lookup required geomagnetic/solar wind conditions for that period. Run the diffusion solver for that setup and plot the results.

\texttt{>>> rb.evolve()}

\texttt{>>> rb.plot()}



 %%%%%%%%%%%%%%%%%%%%%%%%
\section{OMNI Module}

bla bla

 %%%%%%%%%%%%%%%%%%%%%%%%
\section{ONERA-DESP Module}

bla bla


  %%%%%%%%%%%%%%%%%%%%%%%%
\section{The testing.py module}

Is supposed to test the implementation of spacepy modules.





\end{document}  